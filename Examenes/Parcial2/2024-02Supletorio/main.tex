\documentclass[12pt, a4paper]{exam}
\usepackage{graphicx}
\usepackage[left=0.2in, top=0.2in, right=0.2in]{geometry}
\usepackage[normalem]{ulem}
\renewcommand\ULthickness{1.0pt}   %%---> For changing thickness of underline
\setlength\ULdepth{1.3ex}%\maxdimen ---> For changing depth of underline
\usepackage{textcomp}
\usepackage{amsmath}
\usepackage{bm}

% document version
\newcommand{\docver}{\input{version}}

% predefined sets
\newcommand{\bools}{\mathbb{B}}
\newcommand{\nats}{\mathbb{N}}
\newcommand{\ints}{\mathbb{Z}}
\newcommand{\rats}{\mathbb{Q}}
\newcommand{\reals}{\mathbb{R}}
\newcommand{\mts}{\{\}}

% fonts for several objects
\newcommand{\fsys}[1]{\mathsf{#1}}
\newcommand{\fset}[1]{\mathtt{#1}}
\newcommand{\fval}[1]{\mathbf{#1}}
\newcommand{\fsub}[1]{{#1}}
\newcommand{\frlnm}[1]{{\sc #1}}
\newcommand{\extfval}[1]{\overline{\fval{#1}}}
\newcommand{\extfsub}[1]{\overline{\fsub{#1}}}
\newcommand{\fsett}[1]{\mathcal{#1}}
\newcommand{\ff}[1]{\mathit{#1}}
\newcommand{\fp}[1]{\mathit{#1}}
\newcommand{\sort}[2]{{#1}{:}{#2}}

% abbreviations
\newcommand{\fsF}{\fsys{F}}
\newcommand{\DS}{\fsys{DS}}
\newcommand{\DSL}{{\fsys{DS}(\mathcal{L})}}
\newcommand{\LARR}{\mathcal{L}_A}

% Boolean elements
\newcommand{\eF}{\fset{F}}
\newcommand{\eT}{\fset{T}}

% global sets
\newcommand{\vprop}{\fsett{V}}
\newcommand{\prop}{\fsett{T}(\vprop)}
\newcommand{\vars}{\fsett{X}}
\newcommand{\funcs}{\fsett{F}}
\newcommand{\preds}{\fsett{P}}
\newcommand{\lang}{\fsett{L}}
\newcommand{\arity}{\textit{ar}}
\newcommand{\terms}{\fsett{T}_\funcs(\vars)}
\newcommand{\forms}{\fsett{T}_{(\funcs,\preds)}(\vars)}
\newcommand{\srest}[2]{{#2}_{\triangleleft {#1}}}

% logical connectives
\newcommand{\STRUE}{\mathit{true}}
\newcommand{\SFALSE}{\mathit{false}}
\newcommand{\SIFF}{\equiv}
\newcommand{\SXOR}{\not\equiv}
\newcommand{\SOR}{\lor}
\newcommand{\SAND}{\land}
\newcommand{\SNEG}{\neg}
\newcommand{\SIMP}{\rightarrow}
\newcommand{\SCON}{\leftarrow}
\newcommand{\SALL}{\forall}
\newcommand{\SEX}{\exists}

% macros for proposiciones
\newcommand{\TRUE}{\STRUE}
\newcommand{\FALSE}{\SFALSE}
\newcommand{\IFF}[2]{(#1 \SIFF #2)}
\newcommand{\XOR}[2]{(#1 \SXOR #2)}
\newcommand{\OR}[2]{(#1 \SOR #2)}
\newcommand{\AND}[2]{(#1 \SAND #2)}
\newcommand{\NEG}[1]{({\SNEG}#1)}
\newcommand{\IMP}[2]{(#1 \SIMP #2)}
\newcommand{\CON}[2]{(#1 \SCON #2)}
\newcommand{\ALL}[2]{(\SALL{#1}\,{#2})}
\newcommand{\EX}[2]{(\SEX{#1}\,{#2})}

\newcommand{\AIFF}[2]{#1 \SIFF #2}
\newcommand{\AXOR}[2]{#1 \SXOR #2}
\newcommand{\AOR}[2]{#1 \SOR #2}
\newcommand{\AAND}[2]{#1 \SAND #2}
\newcommand{\ANEG}[1]{{\SNEG}#1}
\newcommand{\AIMP}[2]{#1 \SIMP #2}
\newcommand{\ACON}[2]{#1 \SCON #2}
\newcommand{\AALL}[2]{\SALL{#1}\,{#2}}
\newcommand{\AEX}[2]{\SEX{#1}\,{#2}}
\newcommand{\QALL}[3]{(\SALL{#1}\mid {#2}: {#3})}
\newcommand{\QEX}[3]{(\SEX{#1}\mid {#2}: {#3})}
\newcommand{\QALLS}[2]{\left(\SALL{#1}\mid : {#2}\right)}
\newcommand{\QEXS}[2]{\left(\SEX{#1}\mid : {#2}\right)}


% other syntax
\newcommand{\tsub}[3]{#1\!\left[{#2}:={#3}\right]}
\newcommand{\divs}[2]{{#1}\,{\cdot|}\,{#2}}

% environments
\newenvironment{calc}{\begin{array*}}{\end{array*}}
%\newenvironment{calc}{\begin{align*}}{\end{align*}}
\newcommand{\expr}[1]{ & \; {#1} \\}
\newcommand{\exprnnl}[1]{ & {#1}}
\newcommand{\expl}[2]{#1 & \quad \langle \; \textnormal{#2} \;\rangle \\}


\newcommand{\con}[3]{ #1 \overset{#2}{=} #3}
\newcommand{\oexists}[3]{(\exists {#1} \ | \ #2  : \  #3  )}
\newcommand{\oall}[3]{(\forall {#1} \ | \  #2  : \  #3  )}
\newcommand{\hoare}[3]{\{{#1}\} \{{#2}\} \{{#3}\}}
\usepackage{listings}


\begin{document}
	%\thispagestyle{empty}
	\noindent
	\begin{minipage}[l]{0.1\textwidth}
		\noindent
		\includegraphics[width=1.8\textwidth]{Logosimbolo-uniandes_horizontal.png}
	\end{minipage}
\hfill
\begin{minipage}[c]{0.8\textwidth}
	\begin{center}
		{\large \textbf{Ingeniería de Sistemas y Computación} \par
		\large	Diseño y análisis de algoritmos	\par
		\small  Profesor: Mateo Sanabria Ardila	\par
		\small  Parcial 2 (Supletorio)	\par
		}
	\end{center}
\end{minipage}
\par
\vspace{0.2in}
\noindent
\uline{Fecha de entrega: * Octubre 2024	\hfill  2024-02		\hfill Nota máxima: 100}
\par 
\vspace{0.15in}
% {\small \bfseries 	Attempt any five questions }

\begin{questions}
	\pointsdroppedatright
	\question \textbf{(25 pts)} El algoritmo de \verb|Floyd-Warshall| recibe un grafo y
	devuelve una matriz de distancias mínimas entre todos los pares de nodos. Proponga una
	modificación al algoritmo de \verb|Floyd-Warshall| que permita determinar el camino
	más corto entre dos nodos dados sin cambiar su complejidad. \textbf{Explique} cómo
	modificar el algoritmo y proporcione un ejemplo. Puede mostrar el código modificado o
	usar palabras para explicar la modificación.

	\question \textbf{(25 pts)} El profesor Adam tiene tres hijos que, lamentablemente, no
	se llevan bien. El problema es tan grave que los tres viven en casas separadas y
	asisten a escuelas distintas. Además, se niegan a caminar juntos hacia la escuela y
	cada uno se rehúsa a transitar por cualquier calle por la que el otro haya pasado ese
	día. Los niños no tienen problema en que sus caminos se crucen en una esquina.
	Afortunadamente, tanto las casas como las escuelas están en esquinas, pero más allá de
	eso, el profesor no está seguro de si será posible que los tres puedan asistir a la
	misma escuela. El profesor tiene un mapa de la ciudad. \textbf{Explique} cómo formular
	el problema de determinar si los hijos pueden llegar a las escuelas como un problema
	de flujo máximo.
	\textbf{La explicación debe ser clara y concisa; de lo contrario, se podría perder el punto.}

	\question \textbf{(50 pts)} Proponga una implementación (\texttt{Java,Python}) para la
	función \verb|flatopia| que resuelva un caso del problema de las \textbf{Autopistas}.
	La función recibe como entrada el número de ciudades, las coordenadas de cada ciudad,
	el número de autopistas y, finalmente, una lista de las autopistas ya construidas. La
	función debe resolver el caso, retornando una lista de autopistas que deben
	construirse para conectar todas las ciudades. Si no es necesario construir nuevas
	autopistas, retorne una lista vacía.

	\begin{lstlisting}
	flatopia(number_of_cities: int, cities: list[(int, int)], 
		number_of_highways: int, highways: list[(int, int)]) -> list[(int, int)]
	\end{lstlisting}

	Para la implementación, puede suponer que ya ha implementado la estructura de
	\verb|Union-Find|. Después de construir el grafo, la complejidad de la solución no
	debe superar $O(E \log(E))$; \textbf{de lo contrario, no se validará el punto}.

	

\end{questions}
\newpage
\subsection*{Aeropuertos}

Un catálogo de aerolíneas consiste en una lista de vuelos entre pares de ciudades. Un
viaje puede ser construido secuenciando vuelos. Dos compañías aéreas son equivalentes si
ofrecen conexiones entre los mismos pares de ciudades, es decir, la diferencia entre el
numero de vuelos totales por conexiones es menor o igual a tres. Dados los catálogos de
dos compañías aéreas, determina si son equivalentes o no.

\subsubsection*{Entrada}
La entrada comienza con un único número entero positivo en una línea por sí misma
indicando el número de casos siguientes, cada uno de ellos como se describe a
continuación. Esta línea va seguida de una línea en blanco, y también hay una línea en
blanco entre dos entradas consecutivas. La entrada contiene:
\begin{itemize}
    \item Primera línea: el número N de vuelos en el catálogo de la primera compañía;

    \item N líneas siguientes: dos caracteres del alfabeto en mayúsculas separados por un
    espacio, para los nombres de las ciudades de origen y destino de un vuelo;

    \item Línea N + 2: el número M de vuelos en el catálogo de la segunda compañía;

    \item M líneas siguientes: dos caracteres del alfabeto en mayúsculas separados por un
    espacio, para los nombres de las ciudades de origen y destino de un vuelo.
\end{itemize}

\begin{table}[htbp]
    \centering
    \begin{tabular}{|p{7cm}|p{7cm}|}
        \hline
        \textbf{Entrada} & \textbf{Salida} \\
        \hline
        \begin{tabular}{@{}l@{}}
            1 \\
            6 \\
            A B \\
            B E \\
            A E \\
            C F \\
            E C \\
            D A \\
            7 \\
            A B \\
            D A \\
            E C \\
            C F \\
            D B \\
            B E \\
            D F \\
        \end{tabular} &
        NO \\
        \hline
    \end{tabular}
    \label{tab:input_output}
\end{table}




% \begin{table}[htbp]
%     \centering
%     \caption{Entrada y Salida}
%     \begin{tabular}{|p{5cm}|p{5cm}|}
%         \hline
%         \textbf{Entrada} & \textbf{Salida} \\
%         \hline
%         2 & Caso 1 \\
%         3 4 & 0.250000 \\
%         NYY & Caso 2 \\
%         YNY & 0.795918 \\
%         YYN & 0.341108 \\
%         1 & \\
%         1 3 1 & \\
%         4 7 & \\
%         NYNN & \\
%         YNYN & \\
%         NYNY & \\
%         NNYN & \\
%         2 & \\
%         1 3 10 & \\
%         1 4 10 & \\
%         \hline
%     \end{tabular}
%     \label{tab:entrada_salida}
% \end{table}

\end{document}
