\newpage
\subsection*{Aeropuertos}

Un catálogo de aerolíneas consiste en una lista de vuelos entre pares de ciudades. Un
viaje puede ser construido secuenciando vuelos. Dos compañías aéreas son equivalentes si
ofrecen conexiones entre los mismos pares de ciudades, es decir, la diferencia entre el
numero de vuelos totales por conexiones es menor o igual a tres. Dados los catálogos de
dos compañías aéreas, determina si son equivalentes o no.

\subsubsection*{Entrada}
La entrada comienza con un único número entero positivo en una línea por sí misma
indicando el número de casos siguientes, cada uno de ellos como se describe a
continuación. Esta línea va seguida de una línea en blanco, y también hay una línea en
blanco entre dos entradas consecutivas. La entrada contiene:
\begin{itemize}
    \item Primera línea: el número N de vuelos en el catálogo de la primera compañía;

    \item N líneas siguientes: dos caracteres del alfabeto en mayúsculas separados por un
    espacio, para los nombres de las ciudades de origen y destino de un vuelo;

    \item Línea N + 2: el número M de vuelos en el catálogo de la segunda compañía;

    \item M líneas siguientes: dos caracteres del alfabeto en mayúsculas separados por un
    espacio, para los nombres de las ciudades de origen y destino de un vuelo.
\end{itemize}

\begin{table}[htbp]
    \centering
    \begin{tabular}{|p{7cm}|p{7cm}|}
        \hline
        \textbf{Entrada} & \textbf{Salida} \\
        \hline
        \begin{tabular}{@{}l@{}}
            1 \\
            6 \\
            A B \\
            B E \\
            A E \\
            C F \\
            E C \\
            D A \\
            7 \\
            A B \\
            D A \\
            E C \\
            C F \\
            D B \\
            B E \\
            D F \\
        \end{tabular} &
        NO \\
        \hline
    \end{tabular}
    \label{tab:input_output}
\end{table}




% \begin{table}[htbp]
%     \centering
%     \caption{Entrada y Salida}
%     \begin{tabular}{|p{5cm}|p{5cm}|}
%         \hline
%         \textbf{Entrada} & \textbf{Salida} \\
%         \hline
%         2 & Caso 1 \\
%         3 4 & 0.250000 \\
%         NYY & Caso 2 \\
%         YNY & 0.795918 \\
%         YYN & 0.341108 \\
%         1 & \\
%         1 3 1 & \\
%         4 7 & \\
%         NYNN & \\
%         YNYN & \\
%         NYNY & \\
%         NNYN & \\
%         2 & \\
%         1 3 10 & \\
%         1 4 10 & \\
%         \hline
%     \end{tabular}
%     \label{tab:entrada_salida}
% \end{table}
