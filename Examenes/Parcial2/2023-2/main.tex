\documentclass[12pt, a4paper]{exam}
\usepackage{graphicx}
\usepackage[left=0.2in, top=0.2in, right=0.2in]{geometry}
\usepackage[normalem]{ulem}
\renewcommand\ULthickness{1.0pt}   %%---> For changing thickness of underline
\setlength\ULdepth{1.3ex}%\maxdimen ---> For changing depth of underline
\usepackage{textcomp}
\usepackage{amsmath}
\usepackage{bm}

% document version
\newcommand{\docver}{\input{version}}

% predefined sets
\newcommand{\bools}{\mathbb{B}}
\newcommand{\nats}{\mathbb{N}}
\newcommand{\ints}{\mathbb{Z}}
\newcommand{\rats}{\mathbb{Q}}
\newcommand{\reals}{\mathbb{R}}
\newcommand{\mts}{\{\}}

% fonts for several objects
\newcommand{\fsys}[1]{\mathsf{#1}}
\newcommand{\fset}[1]{\mathtt{#1}}
\newcommand{\fval}[1]{\mathbf{#1}}
\newcommand{\fsub}[1]{{#1}}
\newcommand{\frlnm}[1]{{\sc #1}}
\newcommand{\extfval}[1]{\overline{\fval{#1}}}
\newcommand{\extfsub}[1]{\overline{\fsub{#1}}}
\newcommand{\fsett}[1]{\mathcal{#1}}
\newcommand{\ff}[1]{\mathit{#1}}
\newcommand{\fp}[1]{\mathit{#1}}
\newcommand{\sort}[2]{{#1}{:}{#2}}

% abbreviations
\newcommand{\fsF}{\fsys{F}}
\newcommand{\DS}{\fsys{DS}}
\newcommand{\DSL}{{\fsys{DS}(\mathcal{L})}}
\newcommand{\LARR}{\mathcal{L}_A}

% Boolean elements
\newcommand{\eF}{\fset{F}}
\newcommand{\eT}{\fset{T}}

% global sets
\newcommand{\vprop}{\fsett{V}}
\newcommand{\prop}{\fsett{T}(\vprop)}
\newcommand{\vars}{\fsett{X}}
\newcommand{\funcs}{\fsett{F}}
\newcommand{\preds}{\fsett{P}}
\newcommand{\lang}{\fsett{L}}
\newcommand{\arity}{\textit{ar}}
\newcommand{\terms}{\fsett{T}_\funcs(\vars)}
\newcommand{\forms}{\fsett{T}_{(\funcs,\preds)}(\vars)}
\newcommand{\srest}[2]{{#2}_{\triangleleft {#1}}}

% logical connectives
\newcommand{\STRUE}{\mathit{true}}
\newcommand{\SFALSE}{\mathit{false}}
\newcommand{\SIFF}{\equiv}
\newcommand{\SXOR}{\not\equiv}
\newcommand{\SOR}{\lor}
\newcommand{\SAND}{\land}
\newcommand{\SNEG}{\neg}
\newcommand{\SIMP}{\rightarrow}
\newcommand{\SCON}{\leftarrow}
\newcommand{\SALL}{\forall}
\newcommand{\SEX}{\exists}

% macros for proposiciones
\newcommand{\TRUE}{\STRUE}
\newcommand{\FALSE}{\SFALSE}
\newcommand{\IFF}[2]{(#1 \SIFF #2)}
\newcommand{\XOR}[2]{(#1 \SXOR #2)}
\newcommand{\OR}[2]{(#1 \SOR #2)}
\newcommand{\AND}[2]{(#1 \SAND #2)}
\newcommand{\NEG}[1]{({\SNEG}#1)}
\newcommand{\IMP}[2]{(#1 \SIMP #2)}
\newcommand{\CON}[2]{(#1 \SCON #2)}
\newcommand{\ALL}[2]{(\SALL{#1}\,{#2})}
\newcommand{\EX}[2]{(\SEX{#1}\,{#2})}

\newcommand{\AIFF}[2]{#1 \SIFF #2}
\newcommand{\AXOR}[2]{#1 \SXOR #2}
\newcommand{\AOR}[2]{#1 \SOR #2}
\newcommand{\AAND}[2]{#1 \SAND #2}
\newcommand{\ANEG}[1]{{\SNEG}#1}
\newcommand{\AIMP}[2]{#1 \SIMP #2}
\newcommand{\ACON}[2]{#1 \SCON #2}
\newcommand{\AALL}[2]{\SALL{#1}\,{#2}}
\newcommand{\AEX}[2]{\SEX{#1}\,{#2}}
\newcommand{\QALL}[3]{(\SALL{#1}\mid {#2}: {#3})}
\newcommand{\QEX}[3]{(\SEX{#1}\mid {#2}: {#3})}
\newcommand{\QALLS}[2]{\left(\SALL{#1}\mid : {#2}\right)}
\newcommand{\QEXS}[2]{\left(\SEX{#1}\mid : {#2}\right)}


% other syntax
\newcommand{\tsub}[3]{#1\!\left[{#2}:={#3}\right]}
\newcommand{\divs}[2]{{#1}\,{\cdot|}\,{#2}}

% environments
\newenvironment{calc}{\begin{array*}}{\end{array*}}
%\newenvironment{calc}{\begin{align*}}{\end{align*}}
\newcommand{\expr}[1]{ & \; {#1} \\}
\newcommand{\exprnnl}[1]{ & {#1}}
\newcommand{\expl}[2]{#1 & \quad \langle \; \textnormal{#2} \;\rangle \\}


\newcommand{\con}[3]{ #1 \overset{#2}{=} #3}
\newcommand{\oexists}[3]{(\exists {#1} \ | \ #2  : \  #3  )}
\newcommand{\oall}[3]{(\forall {#1} \ | \  #2  : \  #3  )}
\newcommand{\hoare}[3]{\{{#1}\} \{{#2}\} \{{#3}\}}
\usepackage{listings}


\begin{document}
	%\thispagestyle{empty}
	\noindent
	\begin{minipage}[l]{0.1\textwidth}
		\noindent
		\includegraphics[width=1.8\textwidth]{Logosimbolo-uniandes_horizontal.png}
	\end{minipage}
\hfill
\begin{minipage}[c]{0.8\textwidth}
	\begin{center}
		{\large \textbf{Ingeniería de Sistemas y Computación} \par
		\large	Diseño y análisis de algoritmos	\par
		\small  Profesor: Mateo Sanabria Ardila	\par
		\small  Parcial 2	\par
		}
	\end{center}
\end{minipage}
\par
\vspace{0.2in}
\noindent
\uline{Fecha de entrega: \today 	\hfill  2023-20		\hfill Nota máxima: 60}
\par 
\vspace{0.15in}
% {\small \bfseries 	Attempt any five questions }

\begin{questions}
	\pointsdroppedatright
	\question  \textbf{(15pts)} Suponga que aparte de la capacidad de los caminos la red 
    de flujo se tiene \textbf{capacidad en los nodos}. Es decir, para cada nodo $n$, se tiene un
    limite $l(n)$ que limita cuanto flujo puede pasar por $n$. Proponga como transformar
    una red de flujo $G=(V,E)$ con capacidad en nodos a una red de flujo equivalente 
    $G'=(V',E')$ que no tiene capacidad en los nodos, de tal forma que el flujo maximo
    de $G$ y $G'$ es el mismo. Ademas explique cuantos nodos y caminos tiene $G'$ 


    \question  \textbf{(25pts)}  Para los siguiente se espera una argumentación valida basada
    en la teoria vista en clase/bibliografía del curso. Para la calificación de estos ejercicios
    no hay punto medio, si tiene el punto completo y correcto tiene el punto completo, cero en cualquier
    otro caso.
	\begin{parts}
        \part \textbf{(10pts)} De un ejemplo de un grafo dirigido con pesos negativos para
        el cual Dijsktra produce una respuesta incorrecta, \textbf{justifique por que pasa esto}. 
        \part \textbf{(15pts} Suponga que se  da un grafo dirigido cargado 
        $G=(V,E)$, en el cual los caminos que salen del nodo s pueden tener 
        pesos negativos, mientras que los demás pesos son no negativos, 
        y no existen ciclos de peso negativo. Justifique que el algoritmo de Dijkstra 
        encuentra correctamente los caminos más cortos desde $s$ en este grafo.
    \end{parts}
	\question \textbf{(20pts)} 
    Eres parte de un grupo de $(n)$ guerrilleros armados con dispositivos de comunicación 
    móviles. Tienes información que deseas compartir con todos en el grupo. El problema 
    es que ninguno de ustedes está en el mismo lugar y deben comunicarse solo usando 
    los dispositivos móviles. Lo complicado es que debido a las represarías del estado opresor, sus fondos para
    pagar las facturas de los dispositivos se han agotado, así que deben distribuir la 
    información llamándose entre ustedes de la manera más económica posible. Llamarás a 
    varios camaradas, ellos llamarán a algunos de sus camaradas y así sucesivamente 
    hasta que todos en el grupo hayan recibido la información. Cada uno de ustedes tiene 
    un proveedor de servicios de comunicación satélital encriptada diferente y conoce el costo de la llamada de 
    un compañero A a un compañero B, para todas las posibles combinaciones de A y B. 
    Existen roses entre las diferentes facciones por lo que no todos los camaradas se llevan bien entre sí y algunos nunca llamarán a personas 
    que no les agraden. La tarea es encontrar la secuencia de llamadas más barata posible 
    para que la información se propague desde el líder a los otros $n-1$ miembros del grupo 
    guerrillero.

    Los camaradas de la guerrilla se nombre con los números del 0 al $n-1$. El líder tomo
    el numero $0$. El tiene una libreta donde tiene la información de los camaradas. En cada 
    entrada de esa libreta tiene una linea con tres números naturales de la forma $u,v,w$,
    esta linea significa que el camarada $u$ se puede comunicar con el camarada $v$ con un
    costo $w$.

    \begin{verbatim}
        Camaradas: 4
        Libreta:
            0 1 10
            0 2 10
            1 3 20
            2 3 30
        Repuesta: 40
        --------
        Camaradas: 4
        Libreta:
            0 1 10
            1 2 20
            2 0 30
            2 3 100
        Repuesta: 130
        --------
        Camaradas: 4
        Libreta:
            3 1 4
            3 1 3
            3 2 3
            3 2 6
            1 3 9
            3 1 5
            2 0 9
            0 1 8
            0 2 8
            3 1 2
            2 1 3
            1 3 7
            1 3 5
        Repuesta: 13
    \end{verbatim}

    Usted es líder de la guerrilla, para acabar con el estado opresor debe proponer un algoritmo 
    que dado, el numero de camaradas en la guerrilla y el contenido de la libreta responda
    con el valor del costo mínimo. 

    Debido a que los recursos son limitados, su algoritmo debe correr en tiempo polinomial en el tamaño
    de la guerrilla. Ademas su algoritmo debe estar escrito en algún lenguaje de programación, una 
    explicación de la solución no le sirve a la guerrilla. 

    \textit{Cualquier algoritmo visto en clase debe ser implementado.}

    


 
	

\end{questions}
\end{document}
