\newpage
\subsection*{Mariposas}

Con la llegada de la primavera, los campos se adornan con coloridas flores, atrayendo a
una multitud de mariposas. No todas las especies tienen la misma suerte en encontrar
hábitats adecuados. Como consecuencia, muchas parejas de mariposas ahora habitan prados
separados. Naturalmente, se esfuerzan por reunirse tan a menudo como sea posible. Para
facilitar esto, las mariposas han entrado en negociaciones con los vientos. En lugar de
seguir una ruta fija de un prado a otro, el camino se determina seleccionando un número
entero al azar entre 1 y R inclusive, con todos los números igualmente probables.
Desafortunadamente, este proceso se repite varias veces cuando no hay una corriente de
aire directa entre los prados donde reside una pareja. Esto hace que la distancia total de
un viaje sea bastante impredecible. Ayuda a las parejas de mariposas a determinar la
probabilidad de que una de ellas pueda alcanzar el otro prado. Dado el número de prados y
una lista de corrientes de aire directas, se supone que tu programa debe procesar una
lista de parejas. Para cada pareja, conoces sus capacidades de vuelo y dónde residen. Por
supuesto, siempre elegirán la ruta con la distancia esperada más baja. Tal ruta existe
entre cualquier par de prados.

\subsubsection*{Entrada}

La primera línea contiene el número de casos de prueba que siguen. Cada caso de prueba
comienza con una línea que contiene el número N de prados $(1 \leq N \leq 100)$, seguido de la
distancia máxima R que una mariposa puede volar de una vez $(1 \leq R \leq 30)$. Las siguientes N
líneas contienen N caracteres cada uno. El j-ésimo carácter en la i-ésima línea es "Y" si
hay una corriente de aire directa entre los prados i y j, pero "N" en caso contrario. El
j-ésimo carácter en la i-ésima línea siempre es el mismo que el carácter i-ésimo en la
j-ésima línea. El j-ésimo carácter en la línea j-ésima siempre es "N". Cada caso de prueba
continúa con el número C de pares de mariposas en una línea por sí misma $(1 \leq C \leq 1000)$.
Luego, para cada par, hay una línea que contiene tres enteros a, b y d. Estos números
indican que una de ellas reside en el prado a, la otra en el prado b $(1 \leq a, b \leq N, a \not =
b)$, y la distancia máxima que pueden volar es d $(1 \leq d \leq 10000)$.

\begin{table}[htbp]
    \centering
    \caption{Entrada y Salida}
    \begin{tabular}{|p{5cm}|p{5cm}|}
        \hline
        \textbf{Entrada} & \textbf{Salida} \\
        \hline
        2 & Caso 1 \\
        3 4 & 0.250000 \\
        NYY & Caso 2 \\
        YNY & 0.795918 \\
        YYN & 0.341108 \\
        1 & \\
        1 3 1 & \\
        4 7 & \\
        NYNN & \\
        YNYN & \\
        NYNY & \\
        NNYN & \\
        2 & \\
        1 3 10 & \\
        1 4 10 & \\
        \hline
    \end{tabular}
    \label{tab:entrada_salida}
\end{table}
