\documentclass[12pt, a4paper]{exam}
\usepackage{graphicx}
\usepackage[left=0.2in, top=0.2in, right=0.2in]{geometry}
\usepackage[normalem]{ulem}
\renewcommand\ULthickness{1.0pt}   %%---> For changing thickness of underline
\setlength\ULdepth{1.3ex}%\maxdimen ---> For changing depth of underline
\usepackage{textcomp}
\usepackage{amsmath}
\usepackage{bm}

% document version
\newcommand{\docver}{\input{version}}

% predefined sets
\newcommand{\bools}{\mathbb{B}}
\newcommand{\nats}{\mathbb{N}}
\newcommand{\ints}{\mathbb{Z}}
\newcommand{\rats}{\mathbb{Q}}
\newcommand{\reals}{\mathbb{R}}
\newcommand{\mts}{\{\}}

% fonts for several objects
\newcommand{\fsys}[1]{\mathsf{#1}}
\newcommand{\fset}[1]{\mathtt{#1}}
\newcommand{\fval}[1]{\mathbf{#1}}
\newcommand{\fsub}[1]{{#1}}
\newcommand{\frlnm}[1]{{\sc #1}}
\newcommand{\extfval}[1]{\overline{\fval{#1}}}
\newcommand{\extfsub}[1]{\overline{\fsub{#1}}}
\newcommand{\fsett}[1]{\mathcal{#1}}
\newcommand{\ff}[1]{\mathit{#1}}
\newcommand{\fp}[1]{\mathit{#1}}
\newcommand{\sort}[2]{{#1}{:}{#2}}

% abbreviations
\newcommand{\fsF}{\fsys{F}}
\newcommand{\DS}{\fsys{DS}}
\newcommand{\DSL}{{\fsys{DS}(\mathcal{L})}}
\newcommand{\LARR}{\mathcal{L}_A}

% Boolean elements
\newcommand{\eF}{\fset{F}}
\newcommand{\eT}{\fset{T}}

% global sets
\newcommand{\vprop}{\fsett{V}}
\newcommand{\prop}{\fsett{T}(\vprop)}
\newcommand{\vars}{\fsett{X}}
\newcommand{\funcs}{\fsett{F}}
\newcommand{\preds}{\fsett{P}}
\newcommand{\lang}{\fsett{L}}
\newcommand{\arity}{\textit{ar}}
\newcommand{\terms}{\fsett{T}_\funcs(\vars)}
\newcommand{\forms}{\fsett{T}_{(\funcs,\preds)}(\vars)}
\newcommand{\srest}[2]{{#2}_{\triangleleft {#1}}}

% logical connectives
\newcommand{\STRUE}{\mathit{true}}
\newcommand{\SFALSE}{\mathit{false}}
\newcommand{\SIFF}{\equiv}
\newcommand{\SXOR}{\not\equiv}
\newcommand{\SOR}{\lor}
\newcommand{\SAND}{\land}
\newcommand{\SNEG}{\neg}
\newcommand{\SIMP}{\rightarrow}
\newcommand{\SCON}{\leftarrow}
\newcommand{\SALL}{\forall}
\newcommand{\SEX}{\exists}

% macros for proposiciones
\newcommand{\TRUE}{\STRUE}
\newcommand{\FALSE}{\SFALSE}
\newcommand{\IFF}[2]{(#1 \SIFF #2)}
\newcommand{\XOR}[2]{(#1 \SXOR #2)}
\newcommand{\OR}[2]{(#1 \SOR #2)}
\newcommand{\AND}[2]{(#1 \SAND #2)}
\newcommand{\NEG}[1]{({\SNEG}#1)}
\newcommand{\IMP}[2]{(#1 \SIMP #2)}
\newcommand{\CON}[2]{(#1 \SCON #2)}
\newcommand{\ALL}[2]{(\SALL{#1}\,{#2})}
\newcommand{\EX}[2]{(\SEX{#1}\,{#2})}

\newcommand{\AIFF}[2]{#1 \SIFF #2}
\newcommand{\AXOR}[2]{#1 \SXOR #2}
\newcommand{\AOR}[2]{#1 \SOR #2}
\newcommand{\AAND}[2]{#1 \SAND #2}
\newcommand{\ANEG}[1]{{\SNEG}#1}
\newcommand{\AIMP}[2]{#1 \SIMP #2}
\newcommand{\ACON}[2]{#1 \SCON #2}
\newcommand{\AALL}[2]{\SALL{#1}\,{#2}}
\newcommand{\AEX}[2]{\SEX{#1}\,{#2}}
\newcommand{\QALL}[3]{(\SALL{#1}\mid {#2}: {#3})}
\newcommand{\QEX}[3]{(\SEX{#1}\mid {#2}: {#3})}
\newcommand{\QALLS}[2]{\left(\SALL{#1}\mid : {#2}\right)}
\newcommand{\QEXS}[2]{\left(\SEX{#1}\mid : {#2}\right)}


% other syntax
\newcommand{\tsub}[3]{#1\!\left[{#2}:={#3}\right]}
\newcommand{\divs}[2]{{#1}\,{\cdot|}\,{#2}}

% environments
\newenvironment{calc}{\begin{array*}}{\end{array*}}
%\newenvironment{calc}{\begin{align*}}{\end{align*}}
\newcommand{\expr}[1]{ & \; {#1} \\}
\newcommand{\exprnnl}[1]{ & {#1}}
\newcommand{\expl}[2]{#1 & \quad \langle \; \textnormal{#2} \;\rangle \\}


\newcommand{\con}[3]{ #1 \overset{#2}{=} #3}
\newcommand{\oexists}[3]{(\exists {#1} \ | \ #2  : \  #3  )}
\newcommand{\oall}[3]{(\forall {#1} \ | \  #2  : \  #3  )}
\newcommand{\hoare}[3]{\{{#1}\} \{{#2}\} \{{#3}\}}
\usepackage{listings}


\begin{document}
	%\thispagestyle{empty}
	\noindent
	\begin{minipage}[l]{0.1\textwidth}
		\noindent
		\includegraphics[width=1.8\textwidth]{Logosimbolo-uniandes_horizontal.png}
	\end{minipage}
\hfill
\begin{minipage}[c]{0.8\textwidth}
	\begin{center}
		{\large \textbf{Ingeniería de Sistemas y Computación} \par
		\large	Diseño y análisis de algoritmos	\par
		\small  Profesor: Mateo Sanabria Ardila	\par
		\small  Parcial 2	\par
		}
	\end{center}
\end{minipage}
\par
\vspace{0.2in}
\noindent
\uline{Fecha de entrega: 15 Abril 2014	\hfill  2024-01		\hfill Nota máxima: 50}
\par 
\vspace{0.15in}
% {\small \bfseries 	Attempt any five questions }

\begin{questions}
	\pointsdroppedatright
	\question \textbf{(15pts)}  El árbol de maxima expansion (MET) es un árbol de
	expansion con un peso mayor o igual que el peso de cualquier otro árbol de expansion.
	\textbf{describa} (y justifique porque es valido) un algoritmo que reciba un grafo
	cargado completamente conectado y retorne su correspondiente MET. Complejidad
	esperada: \textbf{$\mathcal{O}(E \times log \ V)$}.

	\question  \textbf{(15pts)} Suponga que aparte de la capacidad de los caminos la red
    de flujo se tiene \textbf{capacidad en los nodos}. Es decir, para cada nodo $n$, se
    tiene un limite $l(n)$ que limita cuanto flujo puede pasar por $n$. Proponga como
    transformar una red de flujo $G=(V,E)$ con capacidad en nodos a una red de flujo
    equivalente $G'=(V',E')$ que no tiene capacidad en los nodos, de tal forma que el
    flujo máximo de $G$ y $G'$ es el mismo. Ademas de una formula para la cantidad de
    nodos y caminos de $G'$ 


	\question \textbf{(20pts)} Solucione el problema \textbf{Mariposas}  
	usando algún algoritmo visto en clase, puede modificarlo de ser necesario, de no ser
	asi el punto no es valido. Se describe I/O para ejemplificar, si embargo puede asumir
	que tiene un función auxiliar que le hace la carga de datos, Si decide hacer esto debe
	estar claro el funcionamiento de dicha función (i.e. entradas y salidas).  Puede
	usar comentarios para hacer aclaraciones. Presente un \textbf{algoritmo (Java,
	Python...)} que solucione el problema, no es valida una solución en palabras. 
	

\end{questions}
\newpage
\subsection*{Mariposas}

Con la llegada de la primavera, los campos se adornan con coloridas flores, atrayendo a
una multitud de mariposas. No todas las especies tienen la misma suerte en encontrar
hábitats adecuados. Como consecuencia, muchas parejas de mariposas ahora habitan prados
separados. Naturalmente, se esfuerzan por reunirse tan a menudo como sea posible. Para
facilitar esto, las mariposas han entrado en negociaciones con los vientos. En lugar de
seguir una ruta fija de un prado a otro, el camino se determina seleccionando un número
entero al azar entre 1 y R inclusive, con todos los números igualmente probables.
Desafortunadamente, este proceso se repite varias veces cuando no hay una corriente de
aire directa entre los prados donde reside una pareja. Esto hace que la distancia total de
un viaje sea bastante impredecible. Ayuda a las parejas de mariposas a determinar la
probabilidad de que una de ellas pueda alcanzar el otro prado. Dado el número de prados y
una lista de corrientes de aire directas, se supone que tu programa debe procesar una
lista de parejas. Para cada pareja, conoces sus capacidades de vuelo y dónde residen. Por
supuesto, siempre elegirán la ruta con la distancia esperada más baja. Tal ruta existe
entre cualquier par de prados.

\subsubsection*{Entrada}

La primera línea contiene el número de casos de prueba que siguen. Cada caso de prueba
comienza con una línea que contiene el número N de prados $(1 \leq N \leq 100)$, seguido de la
distancia máxima R que una mariposa puede volar de una vez $(1 \leq R \leq 30)$. Las siguientes N
líneas contienen N caracteres cada uno. El j-ésimo carácter en la i-ésima línea es "Y" si
hay una corriente de aire directa entre los prados i y j, pero "N" en caso contrario. El
j-ésimo carácter en la i-ésima línea siempre es el mismo que el carácter i-ésimo en la
j-ésima línea. El j-ésimo carácter en la línea j-ésima siempre es "N". Cada caso de prueba
continúa con el número C de pares de mariposas en una línea por sí misma $(1 \leq C \leq 1000)$.
Luego, para cada par, hay una línea que contiene tres enteros a, b y d. Estos números
indican que una de ellas reside en el prado a, la otra en el prado b $(1 \leq a, b \leq N, a \not =
b)$, y la distancia máxima que pueden volar es d $(1 \leq d \leq 10000)$.

\begin{table}[htbp]
    \centering
    \caption{Entrada y Salida}
    \begin{tabular}{|p{5cm}|p{5cm}|}
        \hline
        \textbf{Entrada} & \textbf{Salida} \\
        \hline
        2 & Caso 1 \\
        3 4 & 0.250000 \\
        NYY & Caso 2 \\
        YNY & 0.795918 \\
        YYN & 0.341108 \\
        1 & \\
        1 3 1 & \\
        4 7 & \\
        NYNN & \\
        YNYN & \\
        NYNY & \\
        NNYN & \\
        2 & \\
        1 3 10 & \\
        1 4 10 & \\
        \hline
    \end{tabular}
    \label{tab:entrada_salida}
\end{table}

\end{document}
