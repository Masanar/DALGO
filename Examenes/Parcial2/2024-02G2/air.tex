\newpage
\subsection*{Aeropuertos}

El gobierno de una nación en desarrollo quiere mejorar el transporte en una de sus áreas
más inaccesibles, en un intento de atraer inversión. La región consiste en varios lugares
importantes que deben tener acceso a un aeropuerto.

Por supuesto, una opción es construir un aeropuerto en cada uno de estos lugares, pero
podría resultar más barato construir menos aeropuertos y conectar el resto de las
ubicaciones mediante carreteras. Dado que estas son carreteras de larga distancia que
conectan importantes ubicaciones en el país (por ejemplo, ciudades, grandes aldeas, áreas
industriales), todas las carreteras son de doble sentido. Además, puede haber más de una
carretera directa posible entre dos áreas. Esto se debe a que puede haber varias maneras
de conectar dos áreas (por ejemplo, una carretera atraviesa un túnel en una montaña
mientras que la otra rodea la montaña, etc.) con costos potencialmente diferentes.

Se considera que una ubicación tiene acceso a un aeropuerto si contiene un aeropuerto o si
es posible viajar por carretera desde allí a otra ubicación que tenga un aeropuerto. Se te
proporciona el costo de construir un aeropuerto y una lista de posibles carreteras entre
pares de ubicaciones y sus costos correspondientes. El gobierno ahora necesita tu ayuda
para decidir la forma más económica de garantizar que cada ubicación tenga acceso a un
aeropuerto. El objetivo es facilitar el acceso a los aeropuertos lo más posible, por lo
que, si existen varias maneras de lograr el costo mínimo, elige la opción que tenga la
mayor cantidad de aeropuertos.

\subsubsection*{Entrada}

La primera línea de entrada contiene el número entero \(T\) (\(T < 25\)), el número de
casos de prueba. El resto de la entrada consiste en \(T\) casos.

Cada caso comienza con tres enteros \(N\), \(M\) y \(A\) (\(0 < N \leq 10,000\), \(0 \leq
M \leq 100,000\), \(0 < A \leq 10,000\)) separados por espacios en blanco. \(N\) es el
número de ubicaciones, \(M\) es el número de posibles carreteras que se pueden construir,
y \(A\) es el costo de construir un aeropuerto.

Las siguientes \(M\) líneas contienen cada una tres enteros \(X\), \(Y\) y \(C\) (\(1 \leq
X, Y \leq N\), \(0 < C \leq 10,000\)), separados por espacios en blanco. \(X\) y \(Y\) son
dos ubicaciones, y \(C\) es el costo de construir una carretera entre \(X\) y \(Y\).

\subsubsection*{Salida}

El programa debe generar exactamente \(T\) líneas, una por cada caso. Cada línea debe
tener la forma ‘Caso \#X: Y Z’, donde \(X\) es el número del caso, \(Y\) es el costo
mínimo de construir carreteras y aeropuertos de manera que todas las ubicaciones tengan
acceso a al menos un aeropuerto, y \(Z\) es el número de aeropuertos que se deben
construir. Como se mencionó antes, si hay varias respuestas con costo mínimo, elige
aquella que maximice el número de aeropuertos.





