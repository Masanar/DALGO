\documentclass[12pt, a4paper]{exam}
\usepackage{graphicx}
\usepackage[margin=0.7in]{geometry}
\usepackage[normalem]{ulem}
\renewcommand\ULthickness{1.0pt}   %%---> For changing thickness of underline
\setlength\ULdepth{1.3ex}%\maxdimen ---> For changing depth of underline
\usepackage{textcomp}
\usepackage{amsmath}
\usepackage{bm}
\usepackage{enumerate}% http://ctan.org/pkg/enumerate
\usepackage{hyperref}
\hypersetup{
    colorlinks=true,
    linkcolor=blue,
    filecolor=magenta,      
    urlcolor=blue,
    }

\urlstyle{same}

% document version
\newcommand{\docver}{\input{version}}

% predefined sets
\newcommand{\bools}{\mathbb{B}}
\newcommand{\nats}{\mathbb{N}}
\newcommand{\ints}{\mathbb{Z}}
\newcommand{\rats}{\mathbb{Q}}
\newcommand{\reals}{\mathbb{R}}
\newcommand{\mts}{\{\}}

% fonts for several objects
\newcommand{\fsys}[1]{\mathsf{#1}}
\newcommand{\fset}[1]{\mathtt{#1}}
\newcommand{\fval}[1]{\mathbf{#1}}
\newcommand{\fsub}[1]{{#1}}
\newcommand{\frlnm}[1]{{\sc #1}}
\newcommand{\extfval}[1]{\overline{\fval{#1}}}
\newcommand{\extfsub}[1]{\overline{\fsub{#1}}}
\newcommand{\fsett}[1]{\mathcal{#1}}
\newcommand{\ff}[1]{\mathit{#1}}
\newcommand{\fp}[1]{\mathit{#1}}
\newcommand{\sort}[2]{{#1}{:}{#2}}

% abbreviations
\newcommand{\fsF}{\fsys{F}}
\newcommand{\DS}{\fsys{DS}}
\newcommand{\DSL}{{\fsys{DS}(\mathcal{L})}}
\newcommand{\LARR}{\mathcal{L}_A}

% Boolean elements
\newcommand{\eF}{\fset{F}}
\newcommand{\eT}{\fset{T}}

% global sets
\newcommand{\vprop}{\fsett{V}}
\newcommand{\prop}{\fsett{T}(\vprop)}
\newcommand{\vars}{\fsett{X}}
\newcommand{\funcs}{\fsett{F}}
\newcommand{\preds}{\fsett{P}}
\newcommand{\lang}{\fsett{L}}
\newcommand{\arity}{\textit{ar}}
\newcommand{\terms}{\fsett{T}_\funcs(\vars)}
\newcommand{\forms}{\fsett{T}_{(\funcs,\preds)}(\vars)}
\newcommand{\srest}[2]{{#2}_{\triangleleft {#1}}}

% logical connectives
\newcommand{\STRUE}{\mathit{true}}
\newcommand{\SFALSE}{\mathit{false}}
\newcommand{\SIFF}{\equiv}
\newcommand{\SXOR}{\not\equiv}
\newcommand{\SOR}{\lor}
\newcommand{\SAND}{\land}
\newcommand{\SNEG}{\neg}
\newcommand{\SIMP}{\rightarrow}
\newcommand{\SCON}{\leftarrow}
\newcommand{\SALL}{\forall}
\newcommand{\SEX}{\exists}

% macros for proposiciones
\newcommand{\TRUE}{\STRUE}
\newcommand{\FALSE}{\SFALSE}
\newcommand{\IFF}[2]{(#1 \SIFF #2)}
\newcommand{\XOR}[2]{(#1 \SXOR #2)}
\newcommand{\OR}[2]{(#1 \SOR #2)}
\newcommand{\AND}[2]{(#1 \SAND #2)}
\newcommand{\NEG}[1]{({\SNEG}#1)}
\newcommand{\IMP}[2]{(#1 \SIMP #2)}
\newcommand{\CON}[2]{(#1 \SCON #2)}
\newcommand{\ALL}[2]{(\SALL{#1}\,{#2})}
\newcommand{\EX}[2]{(\SEX{#1}\,{#2})}

\newcommand{\AIFF}[2]{#1 \SIFF #2}
\newcommand{\AXOR}[2]{#1 \SXOR #2}
\newcommand{\AOR}[2]{#1 \SOR #2}
\newcommand{\AAND}[2]{#1 \SAND #2}
\newcommand{\ANEG}[1]{{\SNEG}#1}
\newcommand{\AIMP}[2]{#1 \SIMP #2}
\newcommand{\ACON}[2]{#1 \SCON #2}
\newcommand{\AALL}[2]{\SALL{#1}\,{#2}}
\newcommand{\AEX}[2]{\SEX{#1}\,{#2}}
\newcommand{\QALL}[3]{(\SALL{#1}\mid {#2}: {#3})}
\newcommand{\QEX}[3]{(\SEX{#1}\mid {#2}: {#3})}
\newcommand{\QALLS}[2]{\left(\SALL{#1}\mid : {#2}\right)}
\newcommand{\QEXS}[2]{\left(\SEX{#1}\mid : {#2}\right)}


% other syntax
\newcommand{\tsub}[3]{#1\!\left[{#2}:={#3}\right]}
\newcommand{\divs}[2]{{#1}\,{\cdot|}\,{#2}}

% environments
\newenvironment{calc}{\begin{array*}}{\end{array*}}
%\newenvironment{calc}{\begin{align*}}{\end{align*}}
\newcommand{\expr}[1]{ & \; {#1} \\}
\newcommand{\exprnnl}[1]{ & {#1}}
\newcommand{\expl}[2]{#1 & \quad \langle \; \textnormal{#2} \;\rangle \\}


\newcommand{\con}[3]{ #1 \overset{#2}{=} #3}
\newcommand{\oexists}[3]{(\exists {#1} \ | \ #2  : \  #3  )}
\newcommand{\oall}[3]{(\forall {#1} \ | \  #2  : \  #3  )}
\newcommand{\hoare}[3]{\{{#1}\} \{{#2}\} \{{#3}\}}
\usepackage{listings}


\begin{document}
%\thispagestyle{empty}
\noindent
\begin{minipage}[l]{0.1\textwidth}
    \noindent
    \includegraphics[width=1.8\textwidth]{Logosimbolo-uniandes_horizontal.png}
\end{minipage}
\hfill
\begin{minipage}[c]{0.8\textwidth}
    \begin{center}
        {\large \textbf{Ingeniería de Sistemas y Computación} \par
            \large	Diseño y análisis de algoritmos	\par
            \small  Profesor: Mateo Sanabria Ardila	\par
            \small  Examen 1	\par
        }
    \end{center}
\end{minipage}
\par
\vspace{0.2in}
\noindent
\uline{Fecha de entrega: 13/Septiembre 	\hfill  2024-20		\hfill Nota máxima: \textbf{100} }
\par
\vspace{0.15in}
% {\small \bfseries 	Attempt any five questions }



\begin{questions}
    \pointsdroppedatright
    \question
    Natalia y Tatiana están trabajando en un nuevo algoritmo de enrutamiento de paquetes,
    para dispositivos IoT en una estación petrolera. Los dos algoritmos solucionan el 
    problema. Sin embargo, debido a las limitaciones de Computación de los dispositivos IoT
    donde se ejecutara el algoritmo deben escoger el algoritmo con la mejor complejidad
    temporal. Para facilitar la tarea del equipo de calidad ellas calculan las ecuaciones
    de recurrencia de los algoritmos y basadas en ello tomaran decision de cual algoritmo 
    elegir:
    \begin{itemize}
        \item Tatianas: $ (n-1)T(n) = (n-2)T(n-1) + n$
        \item Natalias: $ T(n) = 5\sqrt{T(n-1)}$
    \end{itemize}
    
    \begin{parts}
        \part[] \textbf{50pts} Basado en temas vistos en la clase argumente cual 
        de los dos algoritmos debería ser escogido.  
    \end{parts}
    \question
    Mateo es muy malo en matemáticas: suele usar una calculadora, que le funcionó bien
    durante la universidad. Desafortunadamente, ahora está atrapado en una isla desierta
    con su buen amigo Diego tras un accidente de snowboard. Ahora están tratando de pasar
    el tiempo resolviendo algunos problemas interesantes.

    Es un problema muy sencillo: dado un número $N$, ¿cuántas maneras hay de que $K$
    números menores que $N$ y pares sumen $N$?

    Por ejemplo, para $N=20$ y $K=2$ existen 21 maneras: $0+20,1+19,\cdots,20+0$.

    \begin{parts}

        \part[] \textbf{(5pts)} Identifique cuales pueden ser los subproblem repetidos
        para este problema, explique que estructura de datos le puede ayudar a no repetir
        problemas. De un ejemplo, expliquelo y justifique porque esa forma de memoizacion
        subproblemas puede ayudar a solucionar el problema.

        \part[] \textbf{(40pts)} Proponga la implementación de una función \verb|Add(N,K)|
        basada en programación  dinámica que solución el problema. Utilice \verb|Python,Java,...| 
        (\textbf{No es valido respuesta en `palabras'}) para describir la función.
        Complejidad esperada $\mathcal{O}(N^2 K^2)$, no es la mejor
        complejidad para resolver el problema, mas bien entiéndase como una cota superior.
        Si su solución corre en menos de esta cota (y soluciona el problema) su punto
        sera valido de lo contrario el punto completo sera invalido.

        \part[] \textbf{(5pts)}Explique cual es la complejidad temporal de la solución del punto anterior.

    \end{parts}
\end{questions}

\end{document}