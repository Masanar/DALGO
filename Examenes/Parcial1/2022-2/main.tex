\documentclass[12pt, a4paper]{exam}
\usepackage{graphicx}
\usepackage[left=0.8in, top=0.7in, total={6.2in,8in}]{geometry}
\usepackage[normalem]{ulem}
\renewcommand\ULthickness{1.0pt}   %%---> For changing thickness of underline
\setlength\ULdepth{1.3ex}%\maxdimen ---> For changing depth of underline
\usepackage{textcomp}
\usepackage{amsmath}
\usepackage{bm}

% document version
\newcommand{\docver}{\input{version}}

% predefined sets
\newcommand{\bools}{\mathbb{B}}
\newcommand{\nats}{\mathbb{N}}
\newcommand{\ints}{\mathbb{Z}}
\newcommand{\rats}{\mathbb{Q}}
\newcommand{\reals}{\mathbb{R}}
\newcommand{\mts}{\{\}}

% fonts for several objects
\newcommand{\fsys}[1]{\mathsf{#1}}
\newcommand{\fset}[1]{\mathtt{#1}}
\newcommand{\fval}[1]{\mathbf{#1}}
\newcommand{\fsub}[1]{{#1}}
\newcommand{\frlnm}[1]{{\sc #1}}
\newcommand{\extfval}[1]{\overline{\fval{#1}}}
\newcommand{\extfsub}[1]{\overline{\fsub{#1}}}
\newcommand{\fsett}[1]{\mathcal{#1}}
\newcommand{\ff}[1]{\mathit{#1}}
\newcommand{\fp}[1]{\mathit{#1}}
\newcommand{\sort}[2]{{#1}{:}{#2}}

% abbreviations
\newcommand{\fsF}{\fsys{F}}
\newcommand{\DS}{\fsys{DS}}
\newcommand{\DSL}{{\fsys{DS}(\mathcal{L})}}
\newcommand{\LARR}{\mathcal{L}_A}

% Boolean elements
\newcommand{\eF}{\fset{F}}
\newcommand{\eT}{\fset{T}}

% global sets
\newcommand{\vprop}{\fsett{V}}
\newcommand{\prop}{\fsett{T}(\vprop)}
\newcommand{\vars}{\fsett{X}}
\newcommand{\funcs}{\fsett{F}}
\newcommand{\preds}{\fsett{P}}
\newcommand{\lang}{\fsett{L}}
\newcommand{\arity}{\textit{ar}}
\newcommand{\terms}{\fsett{T}_\funcs(\vars)}
\newcommand{\forms}{\fsett{T}_{(\funcs,\preds)}(\vars)}
\newcommand{\srest}[2]{{#2}_{\triangleleft {#1}}}

% logical connectives
\newcommand{\STRUE}{\mathit{true}}
\newcommand{\SFALSE}{\mathit{false}}
\newcommand{\SIFF}{\equiv}
\newcommand{\SXOR}{\not\equiv}
\newcommand{\SOR}{\lor}
\newcommand{\SAND}{\land}
\newcommand{\SNEG}{\neg}
\newcommand{\SIMP}{\rightarrow}
\newcommand{\SCON}{\leftarrow}
\newcommand{\SALL}{\forall}
\newcommand{\SEX}{\exists}

% macros for proposiciones
\newcommand{\TRUE}{\STRUE}
\newcommand{\FALSE}{\SFALSE}
\newcommand{\IFF}[2]{(#1 \SIFF #2)}
\newcommand{\XOR}[2]{(#1 \SXOR #2)}
\newcommand{\OR}[2]{(#1 \SOR #2)}
\newcommand{\AND}[2]{(#1 \SAND #2)}
\newcommand{\NEG}[1]{({\SNEG}#1)}
\newcommand{\IMP}[2]{(#1 \SIMP #2)}
\newcommand{\CON}[2]{(#1 \SCON #2)}
\newcommand{\ALL}[2]{(\SALL{#1}\,{#2})}
\newcommand{\EX}[2]{(\SEX{#1}\,{#2})}

\newcommand{\AIFF}[2]{#1 \SIFF #2}
\newcommand{\AXOR}[2]{#1 \SXOR #2}
\newcommand{\AOR}[2]{#1 \SOR #2}
\newcommand{\AAND}[2]{#1 \SAND #2}
\newcommand{\ANEG}[1]{{\SNEG}#1}
\newcommand{\AIMP}[2]{#1 \SIMP #2}
\newcommand{\ACON}[2]{#1 \SCON #2}
\newcommand{\AALL}[2]{\SALL{#1}\,{#2}}
\newcommand{\AEX}[2]{\SEX{#1}\,{#2}}
\newcommand{\QALL}[3]{(\SALL{#1}\mid {#2}: {#3})}
\newcommand{\QEX}[3]{(\SEX{#1}\mid {#2}: {#3})}
\newcommand{\QALLS}[2]{\left(\SALL{#1}\mid : {#2}\right)}
\newcommand{\QEXS}[2]{\left(\SEX{#1}\mid : {#2}\right)}


% other syntax
\newcommand{\tsub}[3]{#1\!\left[{#2}:={#3}\right]}
\newcommand{\divs}[2]{{#1}\,{\cdot|}\,{#2}}

% environments
\newenvironment{calc}{\begin{array*}}{\end{array*}}
%\newenvironment{calc}{\begin{align*}}{\end{align*}}
\newcommand{\expr}[1]{ & \; {#1} \\}
\newcommand{\exprnnl}[1]{ & {#1}}
\newcommand{\expl}[2]{#1 & \quad \langle \; \textnormal{#2} \;\rangle \\}


\newcommand{\con}[3]{ #1 \overset{#2}{=} #3}
\newcommand{\oexists}[3]{(\exists {#1} \ | \ #2  : \  #3  )}
\newcommand{\oall}[3]{(\forall {#1} \ | \  #2  : \  #3  )}
\newcommand{\hoare}[3]{\{{#1}\} \{{#2}\} \{{#3}\}}
\usepackage{listings}


\begin{document}
	%\thispagestyle{empty}
	\noindent
	\begin{minipage}[l]{0.1\textwidth}
		\noindent
		\includegraphics[width=1.8\textwidth]{Logosimbolo-uniandes_horizontal.png}
	\end{minipage}
\hfill
\begin{minipage}[c]{0.8\textwidth}
	\begin{center}
		{\large \textbf{Ingeniería de Sistemas y Computación} \par
		\large	Diseño y análisis de algoritmos	\par
		\small  Profesor: Mateo Sanabria Ardila	\par
		\small  Parcial 1	\par
		}
	\end{center}
\end{minipage}
\par
\vspace{0.2in}
\noindent
\uline{Fecha de entrega: 09/Sep 	\hfill  2022-20		\hfill Nota máxima: 50}
\par 
\vspace{0.15in}
\noindent
\centering
% {\small \bfseries 	Attempt any five questions }

\begin{questions}
	\pointsdroppedatright
	\question \textbf{(12pts)} Tenga en cuenta la siguiente especificación:
	\begin{lstlisting}
    fun ArrayR (a:array [0,n) of real, i:nat, j:nat)
        var s,m1,m2,k : nat
        {0 <= i <= j < n}
        s:= i + j
        if 
            s mod 2 = 0 -> m1,m2:= s/2 - 1, s/2+1
            [] s mod 2 = 1 -> m1,m2:= (s - 1)/2, (s+1)/2
        fi
        if
            i = j ->  skip
        [] 
            i != j -> ArrayR(a, i, m1);
                         ArrayR(a, m2, j);
                         k:=0
            do i+k <= m1 -> 
                a[i+k],a[m2+k],k:=a[m2+k],a[i+k],k+1
            od
        fi
    \end{lstlisting}
    \begin{parts}
        \part[] \textbf{(3pts)}La función \verb|ArrayR| recibe un arreglo de reales y dos índices \verb|i,j|. Explique que hace esta función.
        \part[] \textbf{(9pts)} Suponga que en la maquina donde se ejecuta \verb|ArrayR| la operación de asignación tiene costo $C_0$, las comparaciones $C_1$, las operaciones multiplicativas (multiplicación, división, modulo) $C_3$ y las operaciones aditivas (suma y resta) $C_4$. Calcule la función de costo.
    \end{parts}
    \newpage
	\question \textbf{(14pts)} Tenga en cuenta la siguiente especificación:
	\begin{lstlisting}
    fun SUM (a:array [0,n) of real, i:nat, j:nat, acc:nat)
        var r : nat
        {0 <= i <= j < n and acc >= 0 }
        if
            i = j -> r := acc;  skip
        [] 
            i != j -> 
                if j mod 2 = 0 -> SUM(a,i,j-1, acc + a[j])
                [] j mod 1 = 1 -> SUM(a,j,j-1, acc)
                fi
        fi
    \end{lstlisting}
    \begin{parts}
        \part[] \textbf{(3pts)} La función \verb|SUM| recibe un arreglo de reales, dos índices \verb|i,j| y numero natural \verb|acc|. Explique que hace esta función.
        \part[] \textbf{(11pts)}Suponga que en la maquina donde se ejecuta \verb|ArrayR| la operación de asignación tiene costo $C_0$, las comparaciones $C_1$, las operaciones multiplicativas (multiplicación, división, modulo) $C_3$ y las operaciones aditivas (suma y resta) $C_4$. Calcule la función de costo y \textbf{calcule al complejidad temporal}.
    \end{parts}
	\question \textbf{(14pts)}Pruebe: 
	\begin{parts} 
	    \part[] \textlbrackdbl \\
	            \quad$\bm{var} \ \ x,y : int;$ \\
	            \quad$\{true\}$\\
	            \quad\quad$x,y := y*y, x*x$ \\
	            \quad\quad$\bm{if} \ \ x \leq y \rightarrow y   :=  y-x$ \\
	            \quad\quad$[] \ y \leq x  \rightarrow x := x-y$ \\
	            \quad\quad$\bm{fi}$ \\
	            \quad$\{ 0 \leq x \wedge 0 \leq y  \}$ \\ \textrbrackdbl
	\end{parts}
	\question \textbf{(10pts)} Describir los datos de entrada, los datos de salida y especificar en forma de predicado de lógica proposicional o lógica de primer orden la precondición y la postcondición para los siguientes problemas:
	\begin{parts}
		\part{} Dado un arreglo $a$ de números naturales, determinar si los elementos de $a$ cumplen que todo índice impar del arreglo $a$ tiene como elemento un numero primo, diferente para cada índice. Todo índice par de $a$ tiene como elemento a un entero de la forma $2^i$ ,para algún i. Además la suma de los elementos de $a$ es un numero de la forma $2^k + 1$, para algún $k$
	\end{parts}
	
        
\end{questions}
\end{document}