\documentclass[12pt, a4paper]{exam}
\usepackage{graphicx}
\usepackage[left=0.5in, top=0.5in, right=0.5in]{geometry}
\usepackage[normalem]{ulem}
\renewcommand\ULthickness{1.0pt}   %%---> For changing thickness of underline
\setlength\ULdepth{1.3ex}%\maxdimen ---> For changing depth of underline
\usepackage{textcomp}
\usepackage{amsmath}
\usepackage{bm}

% document version
\newcommand{\docver}{\input{version}}

% predefined sets
\newcommand{\bools}{\mathbb{B}}
\newcommand{\nats}{\mathbb{N}}
\newcommand{\ints}{\mathbb{Z}}
\newcommand{\rats}{\mathbb{Q}}
\newcommand{\reals}{\mathbb{R}}
\newcommand{\mts}{\{\}}

% fonts for several objects
\newcommand{\fsys}[1]{\mathsf{#1}}
\newcommand{\fset}[1]{\mathtt{#1}}
\newcommand{\fval}[1]{\mathbf{#1}}
\newcommand{\fsub}[1]{{#1}}
\newcommand{\frlnm}[1]{{\sc #1}}
\newcommand{\extfval}[1]{\overline{\fval{#1}}}
\newcommand{\extfsub}[1]{\overline{\fsub{#1}}}
\newcommand{\fsett}[1]{\mathcal{#1}}
\newcommand{\ff}[1]{\mathit{#1}}
\newcommand{\fp}[1]{\mathit{#1}}
\newcommand{\sort}[2]{{#1}{:}{#2}}

% abbreviations
\newcommand{\fsF}{\fsys{F}}
\newcommand{\DS}{\fsys{DS}}
\newcommand{\DSL}{{\fsys{DS}(\mathcal{L})}}
\newcommand{\LARR}{\mathcal{L}_A}

% Boolean elements
\newcommand{\eF}{\fset{F}}
\newcommand{\eT}{\fset{T}}

% global sets
\newcommand{\vprop}{\fsett{V}}
\newcommand{\prop}{\fsett{T}(\vprop)}
\newcommand{\vars}{\fsett{X}}
\newcommand{\funcs}{\fsett{F}}
\newcommand{\preds}{\fsett{P}}
\newcommand{\lang}{\fsett{L}}
\newcommand{\arity}{\textit{ar}}
\newcommand{\terms}{\fsett{T}_\funcs(\vars)}
\newcommand{\forms}{\fsett{T}_{(\funcs,\preds)}(\vars)}
\newcommand{\srest}[2]{{#2}_{\triangleleft {#1}}}

% logical connectives
\newcommand{\STRUE}{\mathit{true}}
\newcommand{\SFALSE}{\mathit{false}}
\newcommand{\SIFF}{\equiv}
\newcommand{\SXOR}{\not\equiv}
\newcommand{\SOR}{\lor}
\newcommand{\SAND}{\land}
\newcommand{\SNEG}{\neg}
\newcommand{\SIMP}{\rightarrow}
\newcommand{\SCON}{\leftarrow}
\newcommand{\SALL}{\forall}
\newcommand{\SEX}{\exists}

% macros for proposiciones
\newcommand{\TRUE}{\STRUE}
\newcommand{\FALSE}{\SFALSE}
\newcommand{\IFF}[2]{(#1 \SIFF #2)}
\newcommand{\XOR}[2]{(#1 \SXOR #2)}
\newcommand{\OR}[2]{(#1 \SOR #2)}
\newcommand{\AND}[2]{(#1 \SAND #2)}
\newcommand{\NEG}[1]{({\SNEG}#1)}
\newcommand{\IMP}[2]{(#1 \SIMP #2)}
\newcommand{\CON}[2]{(#1 \SCON #2)}
\newcommand{\ALL}[2]{(\SALL{#1}\,{#2})}
\newcommand{\EX}[2]{(\SEX{#1}\,{#2})}

\newcommand{\AIFF}[2]{#1 \SIFF #2}
\newcommand{\AXOR}[2]{#1 \SXOR #2}
\newcommand{\AOR}[2]{#1 \SOR #2}
\newcommand{\AAND}[2]{#1 \SAND #2}
\newcommand{\ANEG}[1]{{\SNEG}#1}
\newcommand{\AIMP}[2]{#1 \SIMP #2}
\newcommand{\ACON}[2]{#1 \SCON #2}
\newcommand{\AALL}[2]{\SALL{#1}\,{#2}}
\newcommand{\AEX}[2]{\SEX{#1}\,{#2}}
\newcommand{\QALL}[3]{(\SALL{#1}\mid {#2}: {#3})}
\newcommand{\QEX}[3]{(\SEX{#1}\mid {#2}: {#3})}
\newcommand{\QALLS}[2]{\left(\SALL{#1}\mid : {#2}\right)}
\newcommand{\QEXS}[2]{\left(\SEX{#1}\mid : {#2}\right)}


% other syntax
\newcommand{\tsub}[3]{#1\!\left[{#2}:={#3}\right]}
\newcommand{\divs}[2]{{#1}\,{\cdot|}\,{#2}}

% environments
\newenvironment{calc}{\begin{array*}}{\end{array*}}
%\newenvironment{calc}{\begin{align*}}{\end{align*}}
\newcommand{\expr}[1]{ & \; {#1} \\}
\newcommand{\exprnnl}[1]{ & {#1}}
\newcommand{\expl}[2]{#1 & \quad \langle \; \textnormal{#2} \;\rangle \\}


\newcommand{\con}[3]{ #1 \overset{#2}{=} #3}
\newcommand{\oexists}[3]{(\exists {#1} \ | \ #2  : \  #3  )}
\newcommand{\oall}[3]{(\forall {#1} \ | \  #2  : \  #3  )}
\newcommand{\hoare}[3]{\{{#1}\} \{{#2}\} \{{#3}\}}
\usepackage{listings}


\begin{document}
	%\thispagestyle{empty}
	\noindent
	\begin{minipage}[l]{0.1\textwidth}
		\noindent
		\includegraphics[width=1.8\textwidth]{Logosimbolo-uniandes_horizontal.png}
	\end{minipage}
\hfill
\begin{minipage}[c]{0.8\textwidth}
	\begin{center}
		{\large \textbf{Ingeniería de Sistemas y Computación} \par
		\large	Diseño y análisis de algoritmos	\par
		\small  Profesor: Mateo Sanabria Ardila	\par
		\small  Examen 3	\par
		}
	\end{center}
\end{minipage}
\par
\vspace{0.2in}
\noindent
\uline{Fecha de entrega: Diciembre 02 	\hfill  2024-20		\hfill Nota máxima: 120}
\par 
\vspace{0.15in}
% {\small \bfseries 	Attempt any five questions }
\begin{questions}
    \pointsdroppedatright

    \question \textbf{(+70 pts)} 
    Eres un desarrollador en Facebook, y tu equipo está planeando implementar una nueva 
    funcionalidad para analizar patrones de conectividad entre usuarios. En particular,
    necesitas determinar si toda la red de amigos de un usuario, representada como un 
    grafo \( G_B = (V_B, E_B) \), puede encontrarse como un subconjunto dentro de la red 
    de amigos de otro usuario más grande, representada como un 
    grafo \( G_A = (V_A, E_A) \).

    El problema consiste en decidir si es posible asignar cada persona en \( G_B \) a una
    persona única en \( G_A \) de tal manera que:  
    \begin{itemize}
        \item Cada persona en \( G_B \) se mapea a una persona en \( G_A \).  
        \item Cada amistad entre dos personas en \( G_B \) corresponde a una amistad entre
        sus contrapartes asignadas en \( G_A \).
    \end{itemize}

    Mientras exploras este problema, te das cuenta de que es computacionalmente complicado
    , y tu jefe te solicita justificar por qué ocurre esto. Para esto debes:

    \begin{enumerate}
        \item \textbf{(+5 pts)}  Mostrar el lenguaje aceptado del problema de decisión asociado.
        \item \textbf{(+15 pts)} Mostrar que el problema pertenece a NP.  
        \item \textbf{(+50 pts)} Mostrar que el problema es NP-completo. 
    \end{enumerate}

    Para cualquier algoritmo que proponga debe utilizar pseudocódigo o lenguaje de su 
    preferencia. Una explicación en palabras no sera validad como algoritmo. Cualquier 
    justificación de complejidad debe ser justificada sobre los algoritmos propuestos.
    Especifique las entradas y salidas de los algoritmos propuestos. \textbf{Si cualquiera 
    de los anteriores puntos es incumplido no se otorgará puntaje}.

	\question (\textbf{+20 pst}) Supongamos que generalizamos el problema de la cobertura de
    conjuntos de modo que cada conjunto \(S_i\) en la familia \(F\) tiene un peso asociado \(w_i\) y 
	el peso de una cobertura \(C\) es \(\sum_{S_i \in C} w_i\). La idea es determinar una 
	cobertura de peso mínimo. 

	Muestre cómo generalizar (\textbf{pseudocódigo}) de manera natural 
    \textit{GREEDY-SET-COVER} para proporcionar una solución aproximada a cualquier 
    instancia del problema de cobertura de conjuntos cargados.


    \question \textbf{(+20 pts)} Basado en técnicas vistas en clase, implemente una 
    función (\textbf{CÓDIGO: Python, Java...}) para imprimir todas las secuencias
    válidas (es decir, correctamente abiertas y eventualmente cerradas) de $n$ pares de
    paréntesis.  \textbf{Noy hay puntos intermedios, para tener los puntos completos debe
    proporcionar código que ejecute correctamente la tarea, salvo errores de sintaxis.}
    \begin{verbatim}
    # Muestra por consola todas las secuencias válidas de n pares de paréntesis 
        printParenthesis(n: int) -> None
    \end{verbatim}

\end{questions}


\end{document}
