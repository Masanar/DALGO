\hrule
\section*{Autopistas}

La nación isleña de Flatopia es completamente plana. Desafortunadamente, Flatopia tiene un
sistema de autopistas públicas muy deficiente. El gobierno de Flatopia es consciente de
este problema y ya ha construido una serie de autopistas que conectan algunas de las
ciudades más importantes. Sin embargo, aún hay algunas ciudades a las que no se puede
llegar a través de una autopista. Es necesario construir más autopistas para que sea
posible conducir entre cualquier par de ciudades sin salir del sistema de autopistas.

Las ciudades de Flatopia están numeradas del 1 al \(N\) y la ciudad \(i\) tiene una
posición dada por las coordenadas cartesianas \((x_i, y_i)\). Cada autopista conecta
exactamente dos ciudades. Todas las autopistas (tanto las originales como las que se
construirán) siguen líneas rectas, y por lo tanto, su longitud es igual a la distancia
cartesiana entre las ciudades. Todas las autopistas se pueden usar en ambas direcciones.
Las autopistas pueden cruzarse libremente entre sí, pero un conductor solo puede cambiar
de autopista en una ciudad que esté en el extremo de ambas autopistas.

El gobierno de Flatopia quiere minimizar el costo de construir nuevas autopistas. Sin
embargo, quieren garantizar que cada ciudad sea alcanzable por autopista desde cualquier
otra ciudad. Dado que Flatopia es tan plana, el costo de una autopista es siempre
proporcional a su longitud. Por lo tanto, el sistema de autopistas menos costoso será
aquel que minimice la longitud total de las autopistas nuevas.

\section*{Entrada}

La primera línea de la entrada consiste en un número entero que indica el número de casos
de prueba. Luego hay una línea en blanco, y los conjuntos de datos están separados por una
línea en blanco.

Cada caso de prueba consta de dos partes. La primera parte describe todas las ciudades en
el país, y la segunda parte describe todas las autopistas que ya se han construido.

La primera línea del caso de prueba contiene un único entero \(N\) (\(1 \leq N \leq
750\)), que representa el número de ciudades. Las siguientes \(N\) líneas contienen cada
una dos enteros, \(x_i\) y \(y_i\), separados por un espacio. Estos valores dan las
coordenadas de la ciudad \(i\)-ésima (para \(i\) de 1 a \(N\)). Las coordenadas tendrán un
valor absoluto no mayor a 10000. Cada ciudad tiene una ubicación única.

La siguiente línea contiene un único entero \(M\) (\(0 \leq M \leq 1000\)), que representa
el número de autopistas existentes. Las siguientes \(M\) líneas contienen cada una un par
de enteros separados por un espacio. Estos dos enteros representan un par de números de
ciudades que ya están conectadas por una autopista. Cada par de ciudades está conectado
por a lo sumo una autopista.

\section*{Salida}

Para cada caso de prueba, escribe en el archivo de salida una línea por cada nueva
autopista que debería construirse para conectar todas las ciudades con la menor longitud
total posible de autopistas nuevas. Cada autopista debe ser presentada imprimiendo los
números de las ciudades que conecta, separados por un espacio.

Si no es necesario construir nuevas autopistas (todas las ciudades ya están conectadas),
el archivo de salida debe contener una línea con la frase ‘No new highways need’.  Imprime
una línea en blanco entre los casos de prueba.

\textbf{Ejemplo de Entrada}
\begin{verbatim}
1

9
1 5
0 0
3 2
4 5
5 1
0 4
5 2
1 2
5 3
3
1 3
9 7
1 2
\end{verbatim}

\textbf{Ejemplo de Salida}
\begin{verbatim}
1 6
3 7
4 9
5 7
8 3
\end{verbatim}