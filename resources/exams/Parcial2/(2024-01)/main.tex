\documentclass[12pt, a4paper]{exam}
\usepackage{graphicx}
\usepackage[left=0.2in, top=0.2in, right=0.2in]{geometry}
\usepackage[normalem]{ulem}
\renewcommand\ULthickness{1.0pt}   %%---> For changing thickness of underline
\setlength\ULdepth{1.3ex}%\maxdimen ---> For changing depth of underline
\usepackage{textcomp}
\usepackage{amsmath}
\usepackage{bm}

% document version
\newcommand{\docver}{\input{version}}

% predefined sets
\newcommand{\bools}{\mathbb{B}}
\newcommand{\nats}{\mathbb{N}}
\newcommand{\ints}{\mathbb{Z}}
\newcommand{\rats}{\mathbb{Q}}
\newcommand{\reals}{\mathbb{R}}
\newcommand{\mts}{\{\}}

% fonts for several objects
\newcommand{\fsys}[1]{\mathsf{#1}}
\newcommand{\fset}[1]{\mathtt{#1}}
\newcommand{\fval}[1]{\mathbf{#1}}
\newcommand{\fsub}[1]{{#1}}
\newcommand{\frlnm}[1]{{\sc #1}}
\newcommand{\extfval}[1]{\overline{\fval{#1}}}
\newcommand{\extfsub}[1]{\overline{\fsub{#1}}}
\newcommand{\fsett}[1]{\mathcal{#1}}
\newcommand{\ff}[1]{\mathit{#1}}
\newcommand{\fp}[1]{\mathit{#1}}
\newcommand{\sort}[2]{{#1}{:}{#2}}

% abbreviations
\newcommand{\fsF}{\fsys{F}}
\newcommand{\DS}{\fsys{DS}}
\newcommand{\DSL}{{\fsys{DS}(\mathcal{L})}}
\newcommand{\LARR}{\mathcal{L}_A}

% Boolean elements
\newcommand{\eF}{\fset{F}}
\newcommand{\eT}{\fset{T}}

% global sets
\newcommand{\vprop}{\fsett{V}}
\newcommand{\prop}{\fsett{T}(\vprop)}
\newcommand{\vars}{\fsett{X}}
\newcommand{\funcs}{\fsett{F}}
\newcommand{\preds}{\fsett{P}}
\newcommand{\lang}{\fsett{L}}
\newcommand{\arity}{\textit{ar}}
\newcommand{\terms}{\fsett{T}_\funcs(\vars)}
\newcommand{\forms}{\fsett{T}_{(\funcs,\preds)}(\vars)}
\newcommand{\srest}[2]{{#2}_{\triangleleft {#1}}}

% logical connectives
\newcommand{\STRUE}{\mathit{true}}
\newcommand{\SFALSE}{\mathit{false}}
\newcommand{\SIFF}{\equiv}
\newcommand{\SXOR}{\not\equiv}
\newcommand{\SOR}{\lor}
\newcommand{\SAND}{\land}
\newcommand{\SNEG}{\neg}
\newcommand{\SIMP}{\rightarrow}
\newcommand{\SCON}{\leftarrow}
\newcommand{\SALL}{\forall}
\newcommand{\SEX}{\exists}

% macros for proposiciones
\newcommand{\TRUE}{\STRUE}
\newcommand{\FALSE}{\SFALSE}
\newcommand{\IFF}[2]{(#1 \SIFF #2)}
\newcommand{\XOR}[2]{(#1 \SXOR #2)}
\newcommand{\OR}[2]{(#1 \SOR #2)}
\newcommand{\AND}[2]{(#1 \SAND #2)}
\newcommand{\NEG}[1]{({\SNEG}#1)}
\newcommand{\IMP}[2]{(#1 \SIMP #2)}
\newcommand{\CON}[2]{(#1 \SCON #2)}
\newcommand{\ALL}[2]{(\SALL{#1}\,{#2})}
\newcommand{\EX}[2]{(\SEX{#1}\,{#2})}

\newcommand{\AIFF}[2]{#1 \SIFF #2}
\newcommand{\AXOR}[2]{#1 \SXOR #2}
\newcommand{\AOR}[2]{#1 \SOR #2}
\newcommand{\AAND}[2]{#1 \SAND #2}
\newcommand{\ANEG}[1]{{\SNEG}#1}
\newcommand{\AIMP}[2]{#1 \SIMP #2}
\newcommand{\ACON}[2]{#1 \SCON #2}
\newcommand{\AALL}[2]{\SALL{#1}\,{#2}}
\newcommand{\AEX}[2]{\SEX{#1}\,{#2}}
\newcommand{\QALL}[3]{(\SALL{#1}\mid {#2}: {#3})}
\newcommand{\QEX}[3]{(\SEX{#1}\mid {#2}: {#3})}
\newcommand{\QALLS}[2]{\left(\SALL{#1}\mid : {#2}\right)}
\newcommand{\QEXS}[2]{\left(\SEX{#1}\mid : {#2}\right)}


% other syntax
\newcommand{\tsub}[3]{#1\!\left[{#2}:={#3}\right]}
\newcommand{\divs}[2]{{#1}\,{\cdot|}\,{#2}}

% environments
\newenvironment{calc}{\begin{array*}}{\end{array*}}
%\newenvironment{calc}{\begin{align*}}{\end{align*}}
\newcommand{\expr}[1]{ & \; {#1} \\}
\newcommand{\exprnnl}[1]{ & {#1}}
\newcommand{\expl}[2]{#1 & \quad \langle \; \textnormal{#2} \;\rangle \\}


\newcommand{\con}[3]{ #1 \overset{#2}{=} #3}
\newcommand{\oexists}[3]{(\exists {#1} \ | \ #2  : \  #3  )}
\newcommand{\oall}[3]{(\forall {#1} \ | \  #2  : \  #3  )}
\newcommand{\hoare}[3]{\{{#1}\} \{{#2}\} \{{#3}\}}
\usepackage{listings}


\begin{document}
	%\thispagestyle{empty}
	\noindent
	\begin{minipage}[l]{0.1\textwidth}
		\noindent
		\includegraphics[width=1.8\textwidth]{Logosimbolo-uniandes_horizontal.png}
	\end{minipage}
\hfill
\begin{minipage}[c]{0.8\textwidth}
	\begin{center}
		{\large \textbf{Ingeniería de Sistemas y Computación} \par
		\large	Diseño y análisis de algoritmos	\par
		\small  Profesor: Mateo Sanabria Ardila	\par
		\small  Parcial 2	\par
		}
	\end{center}
\end{minipage}
\par
\vspace{0.2in}
\noindent
\uline{Fecha de entrega: 19 Abril 2024	\hfill  2024-01		\hfill Nota máxima: 50}
\par 
\vspace{0.15in}
% {\small \bfseries 	Attempt any five questions }

\begin{questions}
	\pointsdroppedatright

	\question \textbf{(10pts)} Mateo propone que el algoritmo para encontrar el MST de
	un grafo también permite encontrar el camino de menor peso de entre los nodos del
	mismo. Según esto, dado el MST para encontrar el camino de menor peso entre dos nodos
	se sigue el camino dentro del MST. El algoritmo que propone Mateo es valido? De ser
	valido argumente de lo contrario proponga un contra ejemplo.

	\question  \textbf{(15pts)} Tome solo \textbf{uno} de los siguientes puntos. Se espera
	una argumentación valida, basada en definiciones, teoremas, axiomas,... mencionados en
	la bibliografía acerca de las redes de flujo. \textbf{No se espera una demostración,
	per se. Pero si una argumentación justificada en la teoría.}

	\begin{parts}

        \part Suponga que la red de flujo \verb|G| contiene la arista
        \verb|(u,v)| se crea un nueva red de flujo \verb|G'| creando un nuevo vértice
        \verb|x| y sustituyendo \verb|(u,v)| por nuevas aristas \verb|(u,x)| y
        \verb|(x,v)| tal que \verb|c(u,x)=c(x,v)=c(u,v)|. Demuestre que un flujo máximo en
        \verb|G'| tiene el mismo valor que un flujo máximo en \verb|G|. 

		\part Suponga que una red de flujo \verb|G =(V,E)| no cumple la
        suposición de que, para cualquier nodo $v \in V$, el camino $s \rightarrow v
        \rightarrow t$ se encuentra en la red. Se $u$ un vértice de la red para el cual no
        se cumple que $s \rightarrow u \rightarrow t$. Demuestre que debe existir un flujo
        máximo $f$ tal que $f(u,n)=f(v,n)=0$ para todo $n \in V$.
    \end{parts}


	\question \textbf{(20pts)} Solucione el problema \textbf{Aeropuertos}  
	usando algún algoritmo visto en clase, puede modificarlo de ser necesario, de no ser
	asi el punto no es valido. Se describe I/O para ejemplificar, si embargo puede asumir
	que tiene un función auxiliar que le hace la carga de datos, Si decide hacer esto debe
	estar claro el funcionamiento de dicha función (i.e. entradas y salidas).  Puede
	usar comentarios para hacer aclaraciones. Presente un \textbf{algoritmo (Java,
	Python...)} que solucione el problema, no es valida una solución en palabras. 
	
	\question \textbf{(5pts)} \textbf{Explique} cual es la complejidad temporal de su
	algoritmo solución para el ejercicio anterior.

\end{questions}
\newpage
\subsection*{Aeropuertos}

Un catálogo de aerolíneas consiste en una lista de vuelos entre pares de ciudades. Un
viaje puede ser construido secuenciando vuelos. Dos compañías aéreas son equivalentes si
ofrecen conexiones entre los mismos pares de ciudades, es decir, la diferencia entre el
numero de vuelos totales por conexiones es menor o igual a tres. Dados los catálogos de
dos compañías aéreas, determina si son equivalentes o no.

\subsubsection*{Entrada}
La entrada comienza con un único número entero positivo en una línea por sí misma
indicando el número de casos siguientes, cada uno de ellos como se describe a
continuación. Esta línea va seguida de una línea en blanco, y también hay una línea en
blanco entre dos entradas consecutivas. La entrada contiene:
\begin{itemize}
    \item Primera línea: el número N de vuelos en el catálogo de la primera compañía;

    \item N líneas siguientes: dos caracteres del alfabeto en mayúsculas separados por un
    espacio, para los nombres de las ciudades de origen y destino de un vuelo;

    \item Línea N + 2: el número M de vuelos en el catálogo de la segunda compañía;

    \item M líneas siguientes: dos caracteres del alfabeto en mayúsculas separados por un
    espacio, para los nombres de las ciudades de origen y destino de un vuelo.
\end{itemize}

\begin{table}[htbp]
    \centering
    \begin{tabular}{|p{7cm}|p{7cm}|}
        \hline
        \textbf{Entrada} & \textbf{Salida} \\
        \hline
        \begin{tabular}{@{}l@{}}
            1 \\
            6 \\
            A B \\
            B E \\
            A E \\
            C F \\
            E C \\
            D A \\
            7 \\
            A B \\
            D A \\
            E C \\
            C F \\
            D B \\
            B E \\
            D F \\
        \end{tabular} &
        NO \\
        \hline
    \end{tabular}
    \label{tab:input_output}
\end{table}




% \begin{table}[htbp]
%     \centering
%     \caption{Entrada y Salida}
%     \begin{tabular}{|p{5cm}|p{5cm}|}
%         \hline
%         \textbf{Entrada} & \textbf{Salida} \\
%         \hline
%         2 & Caso 1 \\
%         3 4 & 0.250000 \\
%         NYY & Caso 2 \\
%         YNY & 0.795918 \\
%         YYN & 0.341108 \\
%         1 & \\
%         1 3 1 & \\
%         4 7 & \\
%         NYNN & \\
%         YNYN & \\
%         NYNY & \\
%         NNYN & \\
%         2 & \\
%         1 3 10 & \\
%         1 4 10 & \\
%         \hline
%     \end{tabular}
%     \label{tab:entrada_salida}
% \end{table}

\end{document}
