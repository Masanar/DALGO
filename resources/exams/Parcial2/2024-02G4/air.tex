\newpage
\subsection*{ ¿Queda una segunda opción?}


Jesus, siendo el programador más talentoso de su tiempo, no puede pensar que las cosas sean
tan simples. Recientemente, todos sus vecinos han decidido conectarse entre sí mediante
una red (en realidad, todos quieren compartir una conexión de internet de banda ancha
:-)). Pero él quiere minimizar el costo total del cable necesario, ya que es un poco
exigente con los gastos del proyecto. Por alguna razón desconocida, también quiere que
quede una segunda opción. Es decir, quiere saber el segundo mejor costo (si es que existe
alguno que pueda ser igual al mejor costo) para el proyecto. Estoy seguro de que él es
capaz de resolver el problema. Pero está muy ocupado con sus asuntos privados y seguirá
así. Entonces, es tu turno de demostrar que eres un buen programador. 

\section*{Entrada}

La entrada comienza con un número entero \(t \leq 1000\), que indica el número de casos de
prueba a manejar. Luego siguen \(t\) conjuntos de datos, donde cada conjunto comienza con
un par de enteros \(v\) (\(1 \leq v \leq 100\)) y \(e\) (\(0 \leq e \leq 200\)). \(v\)
denota el número de vecinos y \(e\) denota el número de conexiones directas permitidas
entre ellos. Las siguientes \(e\) líneas contienen la descripción de las conexiones
directas permitidas, donde cada línea tiene el formato ‘inicio fin costo’, donde inicio y
fin son los dos extremos de la conexión y costo es el costo de la conexión. Todas las
conexiones son bidireccionales y puede haber múltiples conexiones entre dos extremos.

\section*{Salida}

Puede haber tres casos en la salida:

\begin{enumerate}
    \item No hay manera de completar la tarea.
    \item Solo hay una manera de completar la tarea.
    \item Hay más de una manera.
\end{enumerate}

Imprime ‘No way’ para el primer caso, ‘No second way’ para el segundo caso y un número
entero \(c\) para el tercer caso, donde \(c\) es el segundo mejor costo. La salida para
cada caso debe comenzar en una nueva línea.





