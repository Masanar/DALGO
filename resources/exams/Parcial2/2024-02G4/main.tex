\documentclass[12pt, a4paper]{exam}
\usepackage{graphicx}
\usepackage[left=0.2in, top=0.2in, right=0.2in]{geometry}
\usepackage[normalem]{ulem}
\renewcommand\ULthickness{1.0pt}   %%---> For changing thickness of underline
\setlength\ULdepth{1.3ex}%\maxdimen ---> For changing depth of underline
\usepackage{textcomp}
\usepackage{amsmath}
\usepackage{bm}

% document version
\newcommand{\docver}{\input{version}}

% predefined sets
\newcommand{\bools}{\mathbb{B}}
\newcommand{\nats}{\mathbb{N}}
\newcommand{\ints}{\mathbb{Z}}
\newcommand{\rats}{\mathbb{Q}}
\newcommand{\reals}{\mathbb{R}}
\newcommand{\mts}{\{\}}

% fonts for several objects
\newcommand{\fsys}[1]{\mathsf{#1}}
\newcommand{\fset}[1]{\mathtt{#1}}
\newcommand{\fval}[1]{\mathbf{#1}}
\newcommand{\fsub}[1]{{#1}}
\newcommand{\frlnm}[1]{{\sc #1}}
\newcommand{\extfval}[1]{\overline{\fval{#1}}}
\newcommand{\extfsub}[1]{\overline{\fsub{#1}}}
\newcommand{\fsett}[1]{\mathcal{#1}}
\newcommand{\ff}[1]{\mathit{#1}}
\newcommand{\fp}[1]{\mathit{#1}}
\newcommand{\sort}[2]{{#1}{:}{#2}}

% abbreviations
\newcommand{\fsF}{\fsys{F}}
\newcommand{\DS}{\fsys{DS}}
\newcommand{\DSL}{{\fsys{DS}(\mathcal{L})}}
\newcommand{\LARR}{\mathcal{L}_A}

% Boolean elements
\newcommand{\eF}{\fset{F}}
\newcommand{\eT}{\fset{T}}

% global sets
\newcommand{\vprop}{\fsett{V}}
\newcommand{\prop}{\fsett{T}(\vprop)}
\newcommand{\vars}{\fsett{X}}
\newcommand{\funcs}{\fsett{F}}
\newcommand{\preds}{\fsett{P}}
\newcommand{\lang}{\fsett{L}}
\newcommand{\arity}{\textit{ar}}
\newcommand{\terms}{\fsett{T}_\funcs(\vars)}
\newcommand{\forms}{\fsett{T}_{(\funcs,\preds)}(\vars)}
\newcommand{\srest}[2]{{#2}_{\triangleleft {#1}}}

% logical connectives
\newcommand{\STRUE}{\mathit{true}}
\newcommand{\SFALSE}{\mathit{false}}
\newcommand{\SIFF}{\equiv}
\newcommand{\SXOR}{\not\equiv}
\newcommand{\SOR}{\lor}
\newcommand{\SAND}{\land}
\newcommand{\SNEG}{\neg}
\newcommand{\SIMP}{\rightarrow}
\newcommand{\SCON}{\leftarrow}
\newcommand{\SALL}{\forall}
\newcommand{\SEX}{\exists}

% macros for proposiciones
\newcommand{\TRUE}{\STRUE}
\newcommand{\FALSE}{\SFALSE}
\newcommand{\IFF}[2]{(#1 \SIFF #2)}
\newcommand{\XOR}[2]{(#1 \SXOR #2)}
\newcommand{\OR}[2]{(#1 \SOR #2)}
\newcommand{\AND}[2]{(#1 \SAND #2)}
\newcommand{\NEG}[1]{({\SNEG}#1)}
\newcommand{\IMP}[2]{(#1 \SIMP #2)}
\newcommand{\CON}[2]{(#1 \SCON #2)}
\newcommand{\ALL}[2]{(\SALL{#1}\,{#2})}
\newcommand{\EX}[2]{(\SEX{#1}\,{#2})}

\newcommand{\AIFF}[2]{#1 \SIFF #2}
\newcommand{\AXOR}[2]{#1 \SXOR #2}
\newcommand{\AOR}[2]{#1 \SOR #2}
\newcommand{\AAND}[2]{#1 \SAND #2}
\newcommand{\ANEG}[1]{{\SNEG}#1}
\newcommand{\AIMP}[2]{#1 \SIMP #2}
\newcommand{\ACON}[2]{#1 \SCON #2}
\newcommand{\AALL}[2]{\SALL{#1}\,{#2}}
\newcommand{\AEX}[2]{\SEX{#1}\,{#2}}
\newcommand{\QALL}[3]{(\SALL{#1}\mid {#2}: {#3})}
\newcommand{\QEX}[3]{(\SEX{#1}\mid {#2}: {#3})}
\newcommand{\QALLS}[2]{\left(\SALL{#1}\mid : {#2}\right)}
\newcommand{\QEXS}[2]{\left(\SEX{#1}\mid : {#2}\right)}


% other syntax
\newcommand{\tsub}[3]{#1\!\left[{#2}:={#3}\right]}
\newcommand{\divs}[2]{{#1}\,{\cdot|}\,{#2}}

% environments
\newenvironment{calc}{\begin{array*}}{\end{array*}}
%\newenvironment{calc}{\begin{align*}}{\end{align*}}
\newcommand{\expr}[1]{ & \; {#1} \\}
\newcommand{\exprnnl}[1]{ & {#1}}
\newcommand{\expl}[2]{#1 & \quad \langle \; \textnormal{#2} \;\rangle \\}


\newcommand{\con}[3]{ #1 \overset{#2}{=} #3}
\newcommand{\oexists}[3]{(\exists {#1} \ | \ #2  : \  #3  )}
\newcommand{\oall}[3]{(\forall {#1} \ | \  #2  : \  #3  )}
\newcommand{\hoare}[3]{\{{#1}\} \{{#2}\} \{{#3}\}}
\usepackage{listings}


\begin{document}
	%\thispagestyle{empty}
	\noindent
	\begin{minipage}[l]{0.1\textwidth}
		\noindent
		\includegraphics[width=1.8\textwidth]{Logosimbolo-uniandes_horizontal.png}
	\end{minipage}
\hfill
\begin{minipage}[c]{0.8\textwidth}
	\begin{center}
		{\large \textbf{Ingeniería de Sistemas y Computación} \par
		\large	Diseño y análisis de algoritmos	\par
		\small  Profesor: Mateo Sanabria Ardila	\par
		\small  Parcial 2	\par
		}
	\end{center}
\end{minipage}
\par
\vspace{0.2in}
\noindent
\uline{Fecha de entrega: 22 Octubre 2024	\hfill  2024-02		\hfill Nota máxima: 100}
\par 
\vspace{0.15in}
% {\small \bfseries 	Attempt any five questions }

\begin{questions}
	\pointsdroppedatright

	\question \textbf{(25pts)} Mateo propone que una forma de saber si un grafo (con pesos
	positivos y sin nodos con conexiones a sí mismos) contiene ciclos es tomando la matriz
	de adyacencia del grafo, reemplazar la diagonal principal por $-1$ y ejecutar el
	algoritmo de Floyd-Warshall. Si en la diagonal de la matriz resultante hay un valor
	diferente de $-1$, es porque el grafo contiene ciclos. Explique por qué esta propuesta
	es correcta o incorrecta. Si es incorrecta, proponga una corrección.

	\question \textbf{(25pts)} El profesor Adam tiene dos hijos que, lamentablemente, no
	se llevan bien. El problema es tan grave que no solo se niegan a caminar juntos hacia
	la escuela, sino que además cada uno se rehúsa a caminar por cualquier calle por la
	que el otro haya pasado ese día. Los niños no tienen problema en que sus caminos se
	crucen en una esquina. Afortunadamente, tanto la casa del profesor como la escuela
	están en esquinas, pero más allá de eso, no está seguro de si será posible que ambos
	vayan a la misma escuela.  El profesor tiene un mapa de su ciudad. \textbf{Explique},
	cómo formular el problema de determinar si ambos hijos pueden ir a la misma escuela
	como un problema de máximo flujo.
	\textbf{Su explicación debe ser clara y concisa, de lo contrario pondrá perder el punto.}



	\question \textbf{(50pts)} Proponga una implementación (\verb|Java,Python|) para la
	función \verb|ways| que soluciona un caso del problema \textbf{¿Queda una segunda
	opción?}. La función recibe como parámetro el grafo (lista de aristas) asociado a la
	red del caso de prueba. La función debe dar solución al caso retornando el string
	correspondiente; inclusive si hay una respuesta numérica, la función debe retornar el
	string adecuado.

	\begin{lstlisting}
	ways(network: list[(int, int, int)]) -> str
	\end{lstlisting}

	Para la implementación puede suponer que ya implemento los algoritmos: \verb|DFS, BFS, Union-Find|, \verb|Floyd-Warshall| y \verb|Edmonds-Karp|. Sin embargo, se espera que
	quede claro cuáles son las entradas y salidas de cada algoritmo que decida utilizar.
	\textbf{Si esto no es claro o es incorrecto, su punto no será válido}. La complejidad
	de la solución no debe ser mayor a $O(EV)$, \textbf{de lo contrario no será válido el
	punto!}.

	
	

\end{questions}
\newpage
\subsection*{Aeropuertos}

Un catálogo de aerolíneas consiste en una lista de vuelos entre pares de ciudades. Un
viaje puede ser construido secuenciando vuelos. Dos compañías aéreas son equivalentes si
ofrecen conexiones entre los mismos pares de ciudades, es decir, la diferencia entre el
numero de vuelos totales por conexiones es menor o igual a tres. Dados los catálogos de
dos compañías aéreas, determina si son equivalentes o no.

\subsubsection*{Entrada}
La entrada comienza con un único número entero positivo en una línea por sí misma
indicando el número de casos siguientes, cada uno de ellos como se describe a
continuación. Esta línea va seguida de una línea en blanco, y también hay una línea en
blanco entre dos entradas consecutivas. La entrada contiene:
\begin{itemize}
    \item Primera línea: el número N de vuelos en el catálogo de la primera compañía;

    \item N líneas siguientes: dos caracteres del alfabeto en mayúsculas separados por un
    espacio, para los nombres de las ciudades de origen y destino de un vuelo;

    \item Línea N + 2: el número M de vuelos en el catálogo de la segunda compañía;

    \item M líneas siguientes: dos caracteres del alfabeto en mayúsculas separados por un
    espacio, para los nombres de las ciudades de origen y destino de un vuelo.
\end{itemize}

\begin{table}[htbp]
    \centering
    \begin{tabular}{|p{7cm}|p{7cm}|}
        \hline
        \textbf{Entrada} & \textbf{Salida} \\
        \hline
        \begin{tabular}{@{}l@{}}
            1 \\
            6 \\
            A B \\
            B E \\
            A E \\
            C F \\
            E C \\
            D A \\
            7 \\
            A B \\
            D A \\
            E C \\
            C F \\
            D B \\
            B E \\
            D F \\
        \end{tabular} &
        NO \\
        \hline
    \end{tabular}
    \label{tab:input_output}
\end{table}




% \begin{table}[htbp]
%     \centering
%     \caption{Entrada y Salida}
%     \begin{tabular}{|p{5cm}|p{5cm}|}
%         \hline
%         \textbf{Entrada} & \textbf{Salida} \\
%         \hline
%         2 & Caso 1 \\
%         3 4 & 0.250000 \\
%         NYY & Caso 2 \\
%         YNY & 0.795918 \\
%         YYN & 0.341108 \\
%         1 & \\
%         1 3 1 & \\
%         4 7 & \\
%         NYNN & \\
%         YNYN & \\
%         NYNY & \\
%         NNYN & \\
%         2 & \\
%         1 3 10 & \\
%         1 4 10 & \\
%         \hline
%     \end{tabular}
%     \label{tab:entrada_salida}
% \end{table}

\end{document}
