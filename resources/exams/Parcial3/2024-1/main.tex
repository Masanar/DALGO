\documentclass[12pt, a4paper]{exam}
\usepackage{graphicx}
\usepackage[left=0.5in, top=0.5in, right=0.5in]{geometry}
\usepackage[normalem]{ulem}
\renewcommand\ULthickness{1.0pt}   %%---> For changing thickness of underline
\setlength\ULdepth{1.3ex}%\maxdimen ---> For changing depth of underline
\usepackage{textcomp}
\usepackage{amsmath}
\usepackage{bm}

% document version
\newcommand{\docver}{\input{version}}

% predefined sets
\newcommand{\bools}{\mathbb{B}}
\newcommand{\nats}{\mathbb{N}}
\newcommand{\ints}{\mathbb{Z}}
\newcommand{\rats}{\mathbb{Q}}
\newcommand{\reals}{\mathbb{R}}
\newcommand{\mts}{\{\}}

% fonts for several objects
\newcommand{\fsys}[1]{\mathsf{#1}}
\newcommand{\fset}[1]{\mathtt{#1}}
\newcommand{\fval}[1]{\mathbf{#1}}
\newcommand{\fsub}[1]{{#1}}
\newcommand{\frlnm}[1]{{\sc #1}}
\newcommand{\extfval}[1]{\overline{\fval{#1}}}
\newcommand{\extfsub}[1]{\overline{\fsub{#1}}}
\newcommand{\fsett}[1]{\mathcal{#1}}
\newcommand{\ff}[1]{\mathit{#1}}
\newcommand{\fp}[1]{\mathit{#1}}
\newcommand{\sort}[2]{{#1}{:}{#2}}

% abbreviations
\newcommand{\fsF}{\fsys{F}}
\newcommand{\DS}{\fsys{DS}}
\newcommand{\DSL}{{\fsys{DS}(\mathcal{L})}}
\newcommand{\LARR}{\mathcal{L}_A}

% Boolean elements
\newcommand{\eF}{\fset{F}}
\newcommand{\eT}{\fset{T}}

% global sets
\newcommand{\vprop}{\fsett{V}}
\newcommand{\prop}{\fsett{T}(\vprop)}
\newcommand{\vars}{\fsett{X}}
\newcommand{\funcs}{\fsett{F}}
\newcommand{\preds}{\fsett{P}}
\newcommand{\lang}{\fsett{L}}
\newcommand{\arity}{\textit{ar}}
\newcommand{\terms}{\fsett{T}_\funcs(\vars)}
\newcommand{\forms}{\fsett{T}_{(\funcs,\preds)}(\vars)}
\newcommand{\srest}[2]{{#2}_{\triangleleft {#1}}}

% logical connectives
\newcommand{\STRUE}{\mathit{true}}
\newcommand{\SFALSE}{\mathit{false}}
\newcommand{\SIFF}{\equiv}
\newcommand{\SXOR}{\not\equiv}
\newcommand{\SOR}{\lor}
\newcommand{\SAND}{\land}
\newcommand{\SNEG}{\neg}
\newcommand{\SIMP}{\rightarrow}
\newcommand{\SCON}{\leftarrow}
\newcommand{\SALL}{\forall}
\newcommand{\SEX}{\exists}

% macros for proposiciones
\newcommand{\TRUE}{\STRUE}
\newcommand{\FALSE}{\SFALSE}
\newcommand{\IFF}[2]{(#1 \SIFF #2)}
\newcommand{\XOR}[2]{(#1 \SXOR #2)}
\newcommand{\OR}[2]{(#1 \SOR #2)}
\newcommand{\AND}[2]{(#1 \SAND #2)}
\newcommand{\NEG}[1]{({\SNEG}#1)}
\newcommand{\IMP}[2]{(#1 \SIMP #2)}
\newcommand{\CON}[2]{(#1 \SCON #2)}
\newcommand{\ALL}[2]{(\SALL{#1}\,{#2})}
\newcommand{\EX}[2]{(\SEX{#1}\,{#2})}

\newcommand{\AIFF}[2]{#1 \SIFF #2}
\newcommand{\AXOR}[2]{#1 \SXOR #2}
\newcommand{\AOR}[2]{#1 \SOR #2}
\newcommand{\AAND}[2]{#1 \SAND #2}
\newcommand{\ANEG}[1]{{\SNEG}#1}
\newcommand{\AIMP}[2]{#1 \SIMP #2}
\newcommand{\ACON}[2]{#1 \SCON #2}
\newcommand{\AALL}[2]{\SALL{#1}\,{#2}}
\newcommand{\AEX}[2]{\SEX{#1}\,{#2}}
\newcommand{\QALL}[3]{(\SALL{#1}\mid {#2}: {#3})}
\newcommand{\QEX}[3]{(\SEX{#1}\mid {#2}: {#3})}
\newcommand{\QALLS}[2]{\left(\SALL{#1}\mid : {#2}\right)}
\newcommand{\QEXS}[2]{\left(\SEX{#1}\mid : {#2}\right)}


% other syntax
\newcommand{\tsub}[3]{#1\!\left[{#2}:={#3}\right]}
\newcommand{\divs}[2]{{#1}\,{\cdot|}\,{#2}}

% environments
\newenvironment{calc}{\begin{array*}}{\end{array*}}
%\newenvironment{calc}{\begin{align*}}{\end{align*}}
\newcommand{\expr}[1]{ & \; {#1} \\}
\newcommand{\exprnnl}[1]{ & {#1}}
\newcommand{\expl}[2]{#1 & \quad \langle \; \textnormal{#2} \;\rangle \\}


\newcommand{\con}[3]{ #1 \overset{#2}{=} #3}
\newcommand{\oexists}[3]{(\exists {#1} \ | \ #2  : \  #3  )}
\newcommand{\oall}[3]{(\forall {#1} \ | \  #2  : \  #3  )}
\newcommand{\hoare}[3]{\{{#1}\} \{{#2}\} \{{#3}\}}
\usepackage{listings}


\begin{document}
	%\thispagestyle{empty}
	\noindent
	\begin{minipage}[l]{0.1\textwidth}
		\noindent
		\includegraphics[width=1.8\textwidth]{Logosimbolo-uniandes_horizontal.png}
	\end{minipage}
\hfill
\begin{minipage}[c]{0.8\textwidth}
	\begin{center}
		{\large \textbf{Ingeniería de Sistemas y Computación} \par
		\large	Diseño y análisis de algoritmos	\par
		\small  Profesor: Mateo Sanabria Ardila	\par
		\small  Parcial 3	\par
		}
	\end{center}
\end{minipage}
\par
\vspace{0.2in}
\noindent
\uline{Fecha de entrega: Junio 01 	\hfill  2024-10		\hfill Nota máxima: 50}
\par 
\vspace{0.15in}
% {\small \bfseries 	Attempt any five questions }
\begin{questions}
    \pointsdroppedatright

    \question \textbf{(+20 pts)} Recuerden que el problema GRAPH-ISOMORPHISM es el problema
    que, dados dos grafos, requiere una función (si existe) que permita traducir un grafo
    en el otro. Dicho en palabras coloquiales: los dos grafos tienen las mismas conexiones
    pero sus nodos se llaman diferente.
    \begin{itemize}
        \item (\textbf{+5 pts}) Describa en detalle el problema de decisión relacionado.

        \item (\textbf{+5 pts}) El problema de decisión está en NP, por lo tanto, existe
        un algoritmo $A(p,c)$ que, dada una pregunta $p$ y un certificado $c$, decide $p$.
        Describa cómo sería un certificado $c$ para este problema.

        \item (\textbf{+10 pts}) Para que el problema esté en NP, $A$ debe correr en
        tiempo polinomial. Describa cómo debería ser este algoritmo. \textbf{No es
        necesario que utilice código, pero les recomiendo que sea extremadamente claro con
        la forma en la que va a describir este algoritmo. Si no es claro puede perder el
        punto}.
    \end{itemize}

    \question \textbf{(+15 pts)} Bill propone la siguiente idea para construir un
    recorrido aproximado del TSP en un grafo con función de costo que satisface la
    desigualdad triangular:
    \begin{enumerate}
        \item Comienza con un ciclo trivial que consiste en un solo nodo elegido
        arbitrariamente.
        \item En cada paso, identifica el nodo $u$ que no está en el ciclo pero cuya
        distancia a cualquier nodo del ciclo es mínima.
        \item Suponga que el nodo en el ciclo que está más cerca de $u$ es el nodo $v$.
        \item Extienda el ciclo para incluir a $u$ insertando $u$ justo después de $v$.
        \item Repita hasta que todos los nodo estén en el ciclo.
    \end{enumerate}

    \textbf{Responda las siguientes preguntas con una argumentación válida (si no hay 
	argumentación, no es válida la respuesta). No hay puntos intermedios:
	 o tienen bien la pregunta o la tienen mal.}
    \begin{itemize}
        \item (\textbf{+8 pts}) ¿Es posible implementar dicho algoritmo en tiempo
        polinomial?
        \item (\textbf{+7 pts}) ¿Realmente el algoritmo da una solución aproximada para
        TSP?  
        % \item (\textbf{+2 pts}) Si fuera un algoritmo aproximado, ¿cuál es la cota de
        % aproximación?
    \end{itemize}

    \question \textbf{(+10 pts)} Basado en técnicas vistas en clase, implemente un
    algoritmo (\textbf{CÓDIGO: Python, Java...}) para imprimir todas las secuencias
    válidas (es decir, correctamente abiertas y eventualmente cerradas) de $n$ pares de
    paréntesis. \textbf{(5pts)} ¿Cual es la complejidad del algoritmo?
\end{questions}


\end{document}
