\documentclass[12pt, a4paper]{exam}
\usepackage{graphicx}
\usepackage[left=0.8in, top=0.8in, right=0.8in]{geometry}
\usepackage[normalem]{ulem}
\renewcommand\ULthickness{1.0pt}   %%---> For changing thickness of underline
\setlength\ULdepth{1.3ex}%\maxdimen ---> For changing depth of underline
\usepackage{textcomp}
\usepackage{amsmath}
\usepackage{bm}

% document version
\newcommand{\docver}{\input{version}}

% predefined sets
\newcommand{\bools}{\mathbb{B}}
\newcommand{\nats}{\mathbb{N}}
\newcommand{\ints}{\mathbb{Z}}
\newcommand{\rats}{\mathbb{Q}}
\newcommand{\reals}{\mathbb{R}}
\newcommand{\mts}{\{\}}

% fonts for several objects
\newcommand{\fsys}[1]{\mathsf{#1}}
\newcommand{\fset}[1]{\mathtt{#1}}
\newcommand{\fval}[1]{\mathbf{#1}}
\newcommand{\fsub}[1]{{#1}}
\newcommand{\frlnm}[1]{{\sc #1}}
\newcommand{\extfval}[1]{\overline{\fval{#1}}}
\newcommand{\extfsub}[1]{\overline{\fsub{#1}}}
\newcommand{\fsett}[1]{\mathcal{#1}}
\newcommand{\ff}[1]{\mathit{#1}}
\newcommand{\fp}[1]{\mathit{#1}}
\newcommand{\sort}[2]{{#1}{:}{#2}}

% abbreviations
\newcommand{\fsF}{\fsys{F}}
\newcommand{\DS}{\fsys{DS}}
\newcommand{\DSL}{{\fsys{DS}(\mathcal{L})}}
\newcommand{\LARR}{\mathcal{L}_A}

% Boolean elements
\newcommand{\eF}{\fset{F}}
\newcommand{\eT}{\fset{T}}

% global sets
\newcommand{\vprop}{\fsett{V}}
\newcommand{\prop}{\fsett{T}(\vprop)}
\newcommand{\vars}{\fsett{X}}
\newcommand{\funcs}{\fsett{F}}
\newcommand{\preds}{\fsett{P}}
\newcommand{\lang}{\fsett{L}}
\newcommand{\arity}{\textit{ar}}
\newcommand{\terms}{\fsett{T}_\funcs(\vars)}
\newcommand{\forms}{\fsett{T}_{(\funcs,\preds)}(\vars)}
\newcommand{\srest}[2]{{#2}_{\triangleleft {#1}}}

% logical connectives
\newcommand{\STRUE}{\mathit{true}}
\newcommand{\SFALSE}{\mathit{false}}
\newcommand{\SIFF}{\equiv}
\newcommand{\SXOR}{\not\equiv}
\newcommand{\SOR}{\lor}
\newcommand{\SAND}{\land}
\newcommand{\SNEG}{\neg}
\newcommand{\SIMP}{\rightarrow}
\newcommand{\SCON}{\leftarrow}
\newcommand{\SALL}{\forall}
\newcommand{\SEX}{\exists}

% macros for proposiciones
\newcommand{\TRUE}{\STRUE}
\newcommand{\FALSE}{\SFALSE}
\newcommand{\IFF}[2]{(#1 \SIFF #2)}
\newcommand{\XOR}[2]{(#1 \SXOR #2)}
\newcommand{\OR}[2]{(#1 \SOR #2)}
\newcommand{\AND}[2]{(#1 \SAND #2)}
\newcommand{\NEG}[1]{({\SNEG}#1)}
\newcommand{\IMP}[2]{(#1 \SIMP #2)}
\newcommand{\CON}[2]{(#1 \SCON #2)}
\newcommand{\ALL}[2]{(\SALL{#1}\,{#2})}
\newcommand{\EX}[2]{(\SEX{#1}\,{#2})}

\newcommand{\AIFF}[2]{#1 \SIFF #2}
\newcommand{\AXOR}[2]{#1 \SXOR #2}
\newcommand{\AOR}[2]{#1 \SOR #2}
\newcommand{\AAND}[2]{#1 \SAND #2}
\newcommand{\ANEG}[1]{{\SNEG}#1}
\newcommand{\AIMP}[2]{#1 \SIMP #2}
\newcommand{\ACON}[2]{#1 \SCON #2}
\newcommand{\AALL}[2]{\SALL{#1}\,{#2}}
\newcommand{\AEX}[2]{\SEX{#1}\,{#2}}
\newcommand{\QALL}[3]{(\SALL{#1}\mid {#2}: {#3})}
\newcommand{\QEX}[3]{(\SEX{#1}\mid {#2}: {#3})}
\newcommand{\QALLS}[2]{\left(\SALL{#1}\mid : {#2}\right)}
\newcommand{\QEXS}[2]{\left(\SEX{#1}\mid : {#2}\right)}


% other syntax
\newcommand{\tsub}[3]{#1\!\left[{#2}:={#3}\right]}
\newcommand{\divs}[2]{{#1}\,{\cdot|}\,{#2}}

% environments
\newenvironment{calc}{\begin{array*}}{\end{array*}}
%\newenvironment{calc}{\begin{align*}}{\end{align*}}
\newcommand{\expr}[1]{ & \; {#1} \\}
\newcommand{\exprnnl}[1]{ & {#1}}
\newcommand{\expl}[2]{#1 & \quad \langle \; \textnormal{#2} \;\rangle \\}


\newcommand{\con}[3]{ #1 \overset{#2}{=} #3}
\newcommand{\oexists}[3]{(\exists {#1} \ | \ #2  : \  #3  )}
\newcommand{\oall}[3]{(\forall {#1} \ | \  #2  : \  #3  )}
\newcommand{\hoare}[3]{\{{#1}\} \{{#2}\} \{{#3}\}}
\usepackage{listings}


\begin{document}
	%\thispagestyle{empty}
	\noindent
	\begin{minipage}[l]{0.1\textwidth}
		\noindent
		\includegraphics[width=1.8\textwidth]{Logosimbolo-uniandes_horizontal.png}
	\end{minipage}
\hfill
\begin{minipage}[c]{0.8\textwidth}
	\begin{center}
		{\large \textbf{Ingeniería de Sistemas y Computación} \par
		\large	Diseño y análisis de algoritmos	\par
		\small  Profesor: Mateo Sanabria Ardila	\par
		\small  Examen 3	Supletorio \par
		}
	\end{center}
\end{minipage}
\par
\vspace{0.2in}
\noindent
\uline{Fecha de entrega: Diciembre 08 	\hfill  2024-20		\hfill Nota máxima: 120}
\par 
\vspace{0.15in}
% {\small \bfseries 	Attempt any five questions }
\begin{questions}
    \pointsdroppedatright

    \question \textbf{(+50 pts)} 
    En una fábrica, un robot debe inspeccionar un conjunto de placas de circuito ubicadas 
    en distintos puntos de una cuadrícula. El robot comienza y termina en una estación de 
    carga y debe visitar cada placa de circuito exactamente una vez. Se proporciona la 
    distancia entre cada par de puntos, y el robot puede desplazarse directamente entre 
    ellos. La configuración de las placas de circuito y la ubicación de la estación de
    carga cambian dinamicamente  debido a la naturaleza de la fábrica. Usted es el 
    ingeniero encargado de programar el robot para que calcule la secuencia de visitas
    que minimice la distancia total recorrida.

    Mientras explorae ste problema, se da cuenta de que es computacionalmente complicado
    , y tu jefe le solicita justificar por qué ocurre esto. Para esto debe:

    \begin{enumerate}
        \item \textbf{(+5 pts)} Mostrar el lenguaje aceptado del problema de decisión asociado.
        \item \textbf{(+15 pts)} Mostrar que el problema pertenece a NP.  
        \item \textbf{(+30 pts)} Mostrar que el problema es NPH. 
    \end{enumerate}

    Para cualquiera de los puntos anteriores que requiera mostrar un algoritmo 
    (o la implementación de una función) debe hacerlo usando pseudocódigo. No es valido 
    una explicación en palabras. Si cree que el problema se relaciona a un problema NPC 
    conocido, no es valido que justifique los puntos mencionando dicho problema, en otras 
    palabras si cree que es un problema conocido debe mostrar NP y NPH para ese problema. 
    Sin embargo, puede usar otros problemas NPC conocidos para demostrar que el problema es NPH.

    \question \textbf{(+20 pts)} Responda las siguientes preguntas, cada pregunta depende 
    de la anterior. Por lo tanto, si la respuesta a una pregunta es incorrecta, las siguientes
    serán incorrectas: 
    \begin{itemize}
        \item (\textbf{+5 pts}) Indique el lenguaje aceptado para el problema.
        \item (\textbf{+5 pts}) Describa el problema de optimización asociado.
        \item (\textbf{+10 pts}) Suponga que existe un algoritmo que resuelve el problema
        el problema de decisión en tiempo constante, muestre un algoritmo (pseudocódigo)
        que resuelva el problema de optimización en tiempo polinomial.
    \end{itemize}

    \question \textbf{(+30 pts)} El Teorema 35.3 ([Cor2009] 35) menciona que si 
    $P \not = NP$ entonces para cualquier constante $1 \leq p$, no existe un algoritmo 
    aproximado en tiempo polinomial  con ratio de aproximación $p$ para el problema 
    generar de TSP. 
	\begin{itemize}
		\item \textbf{(+15 pts)} Muestre como en tiempo polinomial se puede traducir una 
		instancia del problema de TSP en una instancia donde la función de costo satisface 
        la desigualdad triangular. Las dos instancias deben tener el mismo conjunto de tours óptimos. 
		\item \textbf{(+15 pts)} Asuma que $P \not = NP$ explique como el punto anterior 
        no contradice el teorema mencionado. 
	\end{itemize}

    % \newpage

    \question \textbf{(+20 pts BONO)} 
    Dado un total especificado \( t \) y un multiconjunto \( S \) de \( n \) números 
    enteros, encuentra todos los subconjuntos distintos de \( S \) cuyos elementos sumen 
    \( t \). Por ejemplo, si \( t = 4 \) y \( S = \{4, 3, 2, 2, 1, 1\} \), entonces 
    existen cuatro sumas distintas que son iguales a \( t \): \( 4 \), \( 3 + 1 \),
    \( 2 + 2 \), y \( 2 + 1 + 1 \). Un número puede ser usado en una suma hasta la 
    cantidad de veces que aparece en \( S \), y un único número cuenta como una suma.

    Implemente (\textbf{Python, Java...})  la función \textit{printSums}
    utilizando la técnica de \textbf{backtraking} para imprimir todas las posible sumas 
    que se pueden obtener dados $t$ y $S$.

    \textbf{Noy hay puntos intermedios, para tener los puntos 
    completos debe proporcionar código que ejecute correctamente la tarea, 
    salvo errores de sintaxis.}
    
    \begin{verbatim}
# Muestra por consola todas las posibles sumas distintas de los elementos 
# de S que suman t, en cualquier formato y orden.
    printSums(t: int,s: Set) -> None
    \end{verbatim}

\end{questions}


\end{document}
