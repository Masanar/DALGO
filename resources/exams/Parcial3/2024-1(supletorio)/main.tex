\documentclass[12pt, a4paper]{exam}
\usepackage{graphicx}
\usepackage[left=0.8in, top=0.7in, total={6.2in,8in}]{geometry}
\usepackage[normalem]{ulem}
\renewcommand\ULthickness{1.0pt}   %%---> For changing thickness of underline
\setlength\ULdepth{1.3ex}%\maxdimen ---> For changing depth of underline
\usepackage{textcomp}
\usepackage{amsmath}
\usepackage{bm}

% document version
\newcommand{\docver}{\input{version}}

% predefined sets
\newcommand{\bools}{\mathbb{B}}
\newcommand{\nats}{\mathbb{N}}
\newcommand{\ints}{\mathbb{Z}}
\newcommand{\rats}{\mathbb{Q}}
\newcommand{\reals}{\mathbb{R}}
\newcommand{\mts}{\{\}}

% fonts for several objects
\newcommand{\fsys}[1]{\mathsf{#1}}
\newcommand{\fset}[1]{\mathtt{#1}}
\newcommand{\fval}[1]{\mathbf{#1}}
\newcommand{\fsub}[1]{{#1}}
\newcommand{\frlnm}[1]{{\sc #1}}
\newcommand{\extfval}[1]{\overline{\fval{#1}}}
\newcommand{\extfsub}[1]{\overline{\fsub{#1}}}
\newcommand{\fsett}[1]{\mathcal{#1}}
\newcommand{\ff}[1]{\mathit{#1}}
\newcommand{\fp}[1]{\mathit{#1}}
\newcommand{\sort}[2]{{#1}{:}{#2}}

% abbreviations
\newcommand{\fsF}{\fsys{F}}
\newcommand{\DS}{\fsys{DS}}
\newcommand{\DSL}{{\fsys{DS}(\mathcal{L})}}
\newcommand{\LARR}{\mathcal{L}_A}

% Boolean elements
\newcommand{\eF}{\fset{F}}
\newcommand{\eT}{\fset{T}}

% global sets
\newcommand{\vprop}{\fsett{V}}
\newcommand{\prop}{\fsett{T}(\vprop)}
\newcommand{\vars}{\fsett{X}}
\newcommand{\funcs}{\fsett{F}}
\newcommand{\preds}{\fsett{P}}
\newcommand{\lang}{\fsett{L}}
\newcommand{\arity}{\textit{ar}}
\newcommand{\terms}{\fsett{T}_\funcs(\vars)}
\newcommand{\forms}{\fsett{T}_{(\funcs,\preds)}(\vars)}
\newcommand{\srest}[2]{{#2}_{\triangleleft {#1}}}

% logical connectives
\newcommand{\STRUE}{\mathit{true}}
\newcommand{\SFALSE}{\mathit{false}}
\newcommand{\SIFF}{\equiv}
\newcommand{\SXOR}{\not\equiv}
\newcommand{\SOR}{\lor}
\newcommand{\SAND}{\land}
\newcommand{\SNEG}{\neg}
\newcommand{\SIMP}{\rightarrow}
\newcommand{\SCON}{\leftarrow}
\newcommand{\SALL}{\forall}
\newcommand{\SEX}{\exists}

% macros for proposiciones
\newcommand{\TRUE}{\STRUE}
\newcommand{\FALSE}{\SFALSE}
\newcommand{\IFF}[2]{(#1 \SIFF #2)}
\newcommand{\XOR}[2]{(#1 \SXOR #2)}
\newcommand{\OR}[2]{(#1 \SOR #2)}
\newcommand{\AND}[2]{(#1 \SAND #2)}
\newcommand{\NEG}[1]{({\SNEG}#1)}
\newcommand{\IMP}[2]{(#1 \SIMP #2)}
\newcommand{\CON}[2]{(#1 \SCON #2)}
\newcommand{\ALL}[2]{(\SALL{#1}\,{#2})}
\newcommand{\EX}[2]{(\SEX{#1}\,{#2})}

\newcommand{\AIFF}[2]{#1 \SIFF #2}
\newcommand{\AXOR}[2]{#1 \SXOR #2}
\newcommand{\AOR}[2]{#1 \SOR #2}
\newcommand{\AAND}[2]{#1 \SAND #2}
\newcommand{\ANEG}[1]{{\SNEG}#1}
\newcommand{\AIMP}[2]{#1 \SIMP #2}
\newcommand{\ACON}[2]{#1 \SCON #2}
\newcommand{\AALL}[2]{\SALL{#1}\,{#2}}
\newcommand{\AEX}[2]{\SEX{#1}\,{#2}}
\newcommand{\QALL}[3]{(\SALL{#1}\mid {#2}: {#3})}
\newcommand{\QEX}[3]{(\SEX{#1}\mid {#2}: {#3})}
\newcommand{\QALLS}[2]{\left(\SALL{#1}\mid : {#2}\right)}
\newcommand{\QEXS}[2]{\left(\SEX{#1}\mid : {#2}\right)}


% other syntax
\newcommand{\tsub}[3]{#1\!\left[{#2}:={#3}\right]}
\newcommand{\divs}[2]{{#1}\,{\cdot|}\,{#2}}

% environments
\newenvironment{calc}{\begin{array*}}{\end{array*}}
%\newenvironment{calc}{\begin{align*}}{\end{align*}}
\newcommand{\expr}[1]{ & \; {#1} \\}
\newcommand{\exprnnl}[1]{ & {#1}}
\newcommand{\expl}[2]{#1 & \quad \langle \; \textnormal{#2} \;\rangle \\}


\newcommand{\con}[3]{ #1 \overset{#2}{=} #3}
\newcommand{\oexists}[3]{(\exists {#1} \ | \ #2  : \  #3  )}
\newcommand{\oall}[3]{(\forall {#1} \ | \  #2  : \  #3  )}
\newcommand{\hoare}[3]{\{{#1}\} \{{#2}\} \{{#3}\}}
\usepackage{listings}


\begin{document}
	%\thispagestyle{empty}
	\noindent
	\begin{minipage}[l]{0.1\textwidth}
		\noindent
		\includegraphics[width=1.8\textwidth]{Logosimbolo-uniandes_horizontal.png}
	\end{minipage}
\hfill
\begin{minipage}[c]{0.8\textwidth}
	\begin{center}
		{\large \textbf{Ingeniería de Sistemas y Computación} \par
		\large	Diseño y análisis de algoritmos	\par
		\small  Profesor: Mateo Sanabria Ardila	\par
		\small  Parcial supletorio 3	\par
		}
	\end{center}
\end{minipage}
\par
\vspace{0.2in}
\noindent
\uline{Fecha de entrega: \today 	\hfill  2024-10		\hfill Nota máxima: 50}
\par 
\vspace{0.15in}
% {\small \bfseries 	Attempt any five questions }

\begin{questions}
	\pointsdroppedatright
	% \question \textbf{(+5pts)} Demuestre que la reducción polinomial $\leq_P$ es una relación transitiva.
	\question \textbf{(+20pts)} Se sabe que es posible demostrar que TSP es NPC usando el problema HAM-CYLE.
	Para esto se muestra una reduccion polinomial $HAM-CYCLE \leq_p TSP$, y demostrando que TSP
	pertenece a NP. De argumentos basados en la teoría vista en clase para justificar o 
	negar la siguiente afirmación, explore consideraciones y puntos importantes para cualquier caso:
	
		Lo anterior quiere decir que TSP también se puede reducir a HAM-CYCLE y en generar
		cualquier problema NPC se puede reducir  a cualquier problema NPC. Lo mismo se puede 
		decir para los problemas NP.

    \question \textbf{(+16pts)} Se sabe que si $P \not = NP$ entonces para cualquier constante 
	$1 \leq p$, no existe un algoritmo aproximado en tiempo polinomial  con ratio de aproximación 
	$p$ para el problema generar de TSP. 
	\begin{itemize}
		\item \textbf{(+8pts)} Muestre como en tiempo polinomial se puede traducir una 
		instancia del problema de TSP en una instancia donde la función de costo satisface la desigualdad
		triangular. Las dos instancias deben tener el mismo conjunto de tours óptimos. 
		\item \textbf{(+8pts)} Asuma que $P \not = NP$ explique como esto no contradice el teorema mencionado. 
	\end{itemize}
    \question \textbf{(+14pts)} Proponga un algoritmo (\textit{Java,Python,C,...}) basado en alguna heuristica
	vista en clase que solucione el siguiente problema, la entrada del algoritmo debe ser \textit{n:int} 
	y debe mostrar por consola la respuesta adecuada:

	El conjunto $\{1, 2, 3, ..., n\}$ contiene un total de $n!$ permutaciones distintas. 
	Al enlistar y etiquetar todas las permutaciones en orden lexicográfico ascendente, 
	obtenemos la siguiente secuencia para n = 3:
		\[ 123, 132, 213, 231, 312, 321 \]
	Proporcione un algoritmo \textbf{(+10pts)} eficiente que devuelva la k-ésima de las $n!$ permutaciones 
	en esta secuencia, dados los valores de n y k. Para ser eficiente, no debería 
	construir las primeras $k-1$ permutaciones en el proceso. \textbf{(+4pts)} Cual es la complejidad de la solución.
	

\end{questions}
\end{document}
